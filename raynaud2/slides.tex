
%\documentclass[notheorems]{beamer}
%\documentclass[handout]{beamer}
%\documentclass[handout,notes=show]{beamer}

%\usepackage[draft]{pdfcomment}

\usetheme{metropolis}

\usepackage{fontspec}
\usefonttheme{professionalfonts} % using non standard fonts for beamer
\setsansfont[
Path           = /Users/alex/Library/Fonts/,
Extension      = .ttf,
Ligatures      = TeX,
UprightFont    = OpenSans-Light,
BoldFont       = OpenSans-Regular,
ItalicFont     = OpenSans-LightItalic,
BoldItalicFont = OpenSans-Italic
]{OpenSans}
\metroset{block=fill}

%\usecolortheme{dolphin}
% No navigation bars 
\beamertemplatenavigationsymbolsempty

\usepackage{amsmath, amssymb, amsfonts, tikz}
\usepackage[utf8]{inputenc}
\usepackage[T1]{fontenc}
\usepackage[english]{babel}

\newtheorem{proposition}[theorem]{Proposition}
\newtheorem{remarks}[theorem]{Remarks}
\newtheorem{remark}[theorem]{Remark}
\newtheorem{conjecture}[theorem]{Conjecture}


\newcommand{\terminology}[1]{\textbf{#1}}

\newcommand{\pdfnote}[1]{\marginnote{\pdfcomment[icon=note]{#1}}}

\newcommand{\NN}{\mathbf{N}}
\newcommand{\ZZ}{\mathbf{Z}}
\newcommand{\QQ}{\mathbf Q}
\newcommand{\CC}{\mathbf C}
\newcommand{\RR}{\mathbf R}
\newcommand{\FF}{\mathbf F}
\newcommand{\dR}{\mathrm{dR}}
\newcommand{\lt}{<}
\newcommand{\gt}{>}
\newcommand{\amp}{&}
\newcommand{\diff}{\mathop{}\!\mathrm{d}}
\newcommand{\ints}{\mathcal{O}}
\newcommand{\ideal}[1]{\mathfrak{#1}}
\usepackage{mathrsfs}\usepackage{cancel}
\newcommand{\Gal}[2]{\operatorname{Gal}(#1/#2)}
\newcommand{\absgal}[1]{\operatorname{Gal}(\overline{#1}/#1)}
\newcommand{\units}{^{\times}}
\DeclareMathOperator{\Spec}{Spec}
\DeclareMathOperator{\Proj}{Proj}

\DeclareMathOperator{\power}{\mathcal{P}}
\DeclareMathOperator{\aff}{\mathbf{A}}
\DeclareMathOperator{\PP}{\mathbf {P}}
\DeclareMathOperator{\norm}{Norm}
\DeclareMathOperator{\trace}{Tr}
\DeclareMathOperator{\Fr}{Fr}
\DeclareMathOperator{\Frob}{Frob}
\DeclareMathOperator{\NS}{NS}
\DeclareMathOperator{\Der}{Der}
\DeclareMathOperator{\Aut}{Aut}
\DeclareMathOperator{\Out}{Out}
\DeclareMathOperator{\Inn}{Inn}
\DeclareMathOperator{\vf}{\mathcal{V}}
\DeclareMathOperator{\krulldim}{krulldim}
\DeclareMathOperator{\trdeg}{trdeg}
\DeclareMathOperator{\Frac}{Frac}
\DeclareMathOperator{\Prob}{Prob}

\newcommand{\sheaf}[1]{\operatorname{\mathcal{#1}}}
\newcommand{\inv}{^{-1}}
\DeclareMathOperator{\ord}{ord}
\DeclareMathOperator{\divisor}{div}
\DeclareMathOperator{\Hom}{Hom}
\DeclareMathOperator{\coker}{coker}
\newcommand{\pair}[2]{\left\langle #1, #2 \right\rangle}
\DeclareMathOperator{\characteristic}{char}

\newcommand{\lb}{[}
\newcommand{\rb}{]}

\author{Alex J. Best}
\institute{BUNTES}
\date{22/4/2020}
\title{Raynaud's proof III}
\setbeamertemplate{footline}[text line]{\url{https://alexjbest.github.io/talks/raynaud2/slides_h.pdf}}
\begin{document}

\begin{frame}
    \titlepage


\end{frame}

%\begin{frame}
%\frametitle{Table of Contents}
%\tableofcontents[currentsection]
%\end{frame}

\begin{frame}{Overview}
    These slides are available online (in handout form) at

    {\url{https://alexjbest.github.io/talks/raynaud2/slides_h.pdf}}

    \pdfnote{to read along check back definitions, link will be at the bottom of all slides, pls no skip}\pause

    \textbf{Goal:} To explain the proof of case 3) of Raynaud's proof of Abhyankar.

\end{frame}

\begin{frame}{Recall}
    A group is \terminology{rev-$p$} if it appears as an unramified cover of $\aff^1_{\overline\FF_p}$.

    We are proving Abhyankar's conjecture via the following:
    \begin{theorem}
        Let $G$ be a quasi-$p$-group and $S$ a $p$-Sylow subgroup of $G$ then we let $G(S)$ be the subgroup of $G$ generated by all strict quasi-$p$-subgroups of $G$ which have a $p$-Sylow subgroup contained in $S$.

        \begin{enumerate}
            \item If $H$ is a normal $p$-subgroup of $G$ and $G/H$ is rev-$p$ then $G$ is rev-$p$.\pause
            \item If the strict quasi-$p$ subgroups of $G$ are all rev-$p$ then $G(S)$ is rev-$p$.\pause
            \item If $G(S) \ne G$ and if $G$ does not contain a non-trivial normal $p$-subgroup, then $G$ is rev-$p$.\pause \qquad {\huge $\Leftarrow$}
        \end{enumerate}
    \end{theorem}
\end{frame}

\begin{frame}{Example of case 3)}
    Pop-quiz: what is an example of case 3)?\pause

    What about $D_{2\ell}$ for prime $\ell$, this is quasi-2 (and not quasi-$\ell$), the only normal subgroup is $C_\ell$ which is not a $2$-group.

    As $D_{2\ell}$ has $\ell$ distinct subgroups that are isomorphic to $C_2$, each of which is a $2$-Sylow, we have that fixing only one $2$-Sylow $S$ constrains $G(S)$ to be simply $S$ again.

    So $D_{2\ell}$ is an example of case $3)$ for a quasi-2-group.
    %If we consider simple groups that are not $p$-groups we rule out case 1).

   % To find a quasi-$p$-group whose strict quasi-$p$-subgroups do not generate it, we could take one with few strict quasi-$p$-subgroups.

    %What about $A_5$ of order $60$, it's simple, quasi $p$ for $p =2,3,5$, not a $p$-group for any of these. The only strict quasi-$3$ subgroups are the 3-Sylows (3-cycles) and the $A_4$ in $A_5$ (the twisted $S_3$ in $A_5$ is not quasi-$3$), fixing one $3$-Sylow $S$, the $C_3$ is $S$ itself and the conjugates of $A_4$ in $A_5$ which have a $3$-Sylow contained in $S$ but I think this generates!
\end{frame}

\begin{frame}{Overview}
    \begin{enumerate}[<+->]
        \item Construct a curve $Y''$ over a DVR $R$ (with residue field $k$) with an action of $G$ on $Y''$, such that $Y''/G$ has special fibre a tree of $\PP^1$'s.
        \item Associate a graph with a group action to this situation, where irreducible components of $Y''_k$ $\leftrightarrow$ vertices.
        \item (The combinatorial step, next week) Show that a graph with a group action satisfying additional properties must contain a vertex on which the group acts in a specific way.
        \item This vertex corresponds to a component $C$ of $Y''_k$ covering $P$ in $X''_k$ in such a way that the restriction of $C$ to $P - \{ \infty \}$ is etale and Galois of group $G$.
    \end{enumerate}
\end{frame}

\begin{frame}{Semi-stable curves}
    Let $R$ be a discrete valuation ring with fraction field $K$.

    We will outline a general construction that takes $G$-covers between smooth proper $R$-curves $Y \to X$ and creates combinatorially simpler models that can be analysed explicitly. \pause

    \begin{definition}
        Let $Y$ be a smooth projective curve over $K$ with $H^0(Y,\sheaf O_Y)=K$. Then $Y$ is said to be \emph{semi-stable} if there exists a proper model of $Y$ which is at-worst-nodal of relative dimension 1 over $R$ (i.e.\ all closed points of $X_k$ are either in the smooth locus of the structure morphism $X\to\Spec(k)$ or are ordinary double points). We call this a semi-stable model.
    \end{definition}
\end{frame}

\begin{frame}{Inertia graphs}
    Given a semistable model, the special fibre consists of a set of irreducible components linked by double points.

    We can take the dual graph of this set-up, i.e. vertices for irreducible components, with edges connecting the vertices corresponding to a pair of components that meet (this could include self-loops).

    If $G$ acts on a semistable model then we get an action on the corresponding graph.
\end{frame}

\begin{frame}
    \begin{theorem}[Semi-stable reduction theorem]
        Let $X$ be a proper $R$-curve with geometrically connected generic fibre. Then there exists a finite extension $R'$ of $R$, such that there exists a birational and proper $R'$-morphism $\pi  \colon  \widetilde X \to X \times \Spec R'$ where $\widetilde X$ is semi-stable.
    \end{theorem}\pause
    \begin{example}
        Consider the nodal cubic
        $$ y^2 = x^3 + p /\ZZ_p$$
        this does not have semi-stable reduction as on the special fibre the singularity is not an ordinary double point.

        However upon base-extension to $\ZZ_p\lb \sqrt[6]{p}\rb $ we can change the model to get
        $$ y^2 = x^3 + 1 /\ZZ_p$$
        which in fact has good reduction (for $p \ne 2,3$).
    \end{example}
    \end{frame}\begin{frame}
    Let $X$ be an $R$-curve and $x$ a closed point of $X$ such that $X_k$ is reduced at $x$. Then let
    $$ \delta _x = \dim_k \widetilde {\sheaf O}_x/\sheaf O_x $$
    (the normalization of localization of the local ring at $x$ inside its fraction ring).\pause

    Let $m _x$ be the number of maximal ideals of $\sheaf O_x$.\pause
    \end{frame}\begin{frame}{Invariants}
    Then we set
    $$ \mu _x =  2\delta _x - m_x + 1 \in \ZZ_{\ge 0} $$

    which has the property that:\\ $\mu _x = 0 \iff x$ smooth and\\ $\mu _x  = 1 \iff x $ is  an ordinary double point. \pause

    \begin{proposition}
        Let $f\colon Y \to X$  be a covering of $R$-curves with $X_k,Y_k$ both reduced. Let $y$ be a closed point of $Y$ with $x = f(y)$. Then 
        $$ \mu _y \ge \mu _x\text. $$
    \end{proposition}
\end{frame}

\begin{frame}{Local Riemann-Hurwitz}
    \begin{proposition}[Kato]
        Let $f\colon Y \to X$  be a covering of $R$-curves $X_k,Y_k$ both reduced. Let $y$ be a closed point of $Y$ with $x = f(y)$. And $(x_j)_{j\in J}$ the points of the normalization $\widetilde X_k$ over $x$.
        Likewise let $(y_{i,j})_{j\in J, i \in I_j}$ be the points of the normalization of $Y_k$.

        Assume $f_k \colon Y_k \to X_k$ is generically etale. Then
        $$ \mu _y - 1  = n ( \mu _x - 1) + d_K - d_k^w$$
        where $n = \deg (\Spec \hat{\sheaf{O}}_{Y, y} \rightarrow \operatorname{Spec} \hat{\sheaf{O}}_{X, x})$, the value $d_K$ the degree of the ramification divisor of
        $$ \operatorname{Spec}\left(\hat{\sheaf{O}}_{Y, y} \otimes_{R} K\right) \rightarrow \operatorname{Spec}\left(\hat{\sheaf{O}}_{X, x} \otimes_{R} K\right)$$
        and $\displaystyle d_{k}^{w}:=\sum d_{i, j}^{w}, \quad d_{i, j}^{w}:=v_{x_{j}}\left(\delta_{y_{i, j}, x_{j}}\right)-e_{i, j}+1$
    \end{proposition}
\end{frame}
\begin{frame}

    $$ \mu _y - 1  = n ( \mu _x - 1) + d_K - d_k^w$$

    $$\displaystyle d_{k}^{w}:=\sum d_{i, j}^{w}, \quad d_{i, j}^{w}:=v_{x_{j}}\left(\delta_{y_{i, j}, x_{j}}\right)-e_{i, j}+1$$

    where $\delta_{y_{i, j}, x_{j}}$ is the discriminant ideal of the extension $\hat{\sheaf{O}}_{\tilde{X}_{k}, x_{j}} \rightarrow \hat{\sheaf{O}}_{\tilde{Y}_{k}, y_{i, j}}$ of complete DVRs, and $e_{i, j}$ its ramification index. The integer $d_{k}^{w}$ equals 0 if and only if the morphism $\tilde{Y}_{k} \rightarrow \tilde{X}_{k}$ between the normalisations of $X_{k}$ and $Y_{k}$ is tamely ramified.
\end{frame}

\begin{frame}
    \begin{proposition}
        Let $Y$ be an $R$-curve. Let $G$ be a finite group acting by automorphisms on $Y$. Then the quotient $X:=Y / G$ of $Y$ by $G$ exists.
    \end{proposition}
    \begin{proof}
        A quotient exists if and only if every orbit of $G$ is contained in an open affine of $Y$, but as $G$ is finite and every finite set of points in a quasi-projective space (such as $Y$) is contained in an affine.
    \end{proof}
\end{frame}

\begin{frame}
    \begin{proposition}Let $f\colon  Y \rightarrow X$ be a covering. If $Y$ is semi-stable then $X$ is semi-stable.\end{proposition}
    \begin{proof}
        Let $y$ be a closed point of $Y$, and let $x$ be its image in $X$. We have $\mu_{y} \geq \mu_{x}$ and $\mu_{y} \in\{0,1\},$ hence $\mu_{x} \in\{0,1\}$. Thus if $y$ is a smooth point then $x$ is smooth, and if $y$ is an ordinary double point, $x$ is either a double point or a smooth point depending on the number of branches passing through $x$.
    \end{proof}
\end{frame}

\begin{frame}
    \begin{proposition}
        Let $X:=\operatorname{Spec} \sheaf{O}_{X^{\prime}, x}$ be the localisation of an $R$ -curve $X^{\prime}$ at a smooth closed point $x,$ and let $s: S \rightarrow X$ be an $S$ point of $X$. Let $f: Y \rightarrow X$ be a Galois covering, and let $e$ be its ramification index above the point $\tilde{x}:=s(\Spec K)$. Assume that $f$ is étale outside $s(S),$ and that $e$ is prime to $\operatorname{char}(K)$. Then $Y$ is smooth, and the morphism $f_{k}: Y_{k} \rightarrow X_{k}$ is tamely ramified at $x$ with ramification index $e$. In particular the inertia subgroup at a point $y$ of $Y$ above $x$ is cyclic of order $e$.
    \end{proposition}
    \begin{block}{Proof}
        Let $y$ be a closed point of $Y$ above the point $x$. After étale localisation at $y$ and $x$ we can assume that $y$ is the unique closed point of $Y$ which is above $x$. Use local Riemann-Hurwitz.
    \end{block}
\end{frame}
\begin{frame}
    \begin{block}{Proof}
        $$ \mu_{y}-1=n\left(\mu_{x}-1\right)+d_{K}-d_{k}^{w} $$
        We have $\mu_{x}=0$. We compute $d_{K}$. Let $\left\{\tilde{y}_{i}\right\}_{i=1}^{r}$ be the points of $Y_{K}$ above $\tilde{x}$ and let $f$ be the residual degree at these points, then $d_{K}=r(e f-1) .$ Hence $\mu_{y}=1-n+n-r-d_{w}=1-r-d_{w}$. The only possibility is that $r=1, d_{w}=0$ and $\mu_{y}=0$ as claimed. The inertia subgroup at $y$ is then the same as the inertia of the extension $\sheaf{O}_{X_{k}, x} \rightarrow \sheaf{O}_{Y_{k}, y}$ which is cyclic of order $e$ \qedhere.
    \end{block}
    \end{frame}\begin{frame}
    \begin{proposition}
        Assume $\characteristic(K) = 0$ now. Let $\tilde{y}$ be a rational point of $Y_{K}$ which specialises to a point $y$ of $Y_{k}$, and let $\eta$ be the generic point of the irreducible component of $Y_{k}$ containing $y .$ Let $\tilde{x}$ be the image of $\tilde{y}$ in $X_{K}$ and assume that $f_{K}: Y_{K} \rightarrow X_{K}$ is étale outside $\tilde{x} .$ Let $e^{\prime} p^{a}$ be the ramification
        index at $\tilde{y},$ with $e^{\prime}$ prime to $p .$ (Note that $I(\tilde{y})$ and $I(\eta)$ are subgroups of $I(y)$.)
        Then:
        \begin{enumerate}
            \item $I(\eta)$ is a p-group.
            \item $I(\eta)$ is invariant in $I(y),$ and the quotient $I(y) / I(\eta)$ is cyclic of order
        $e^{\prime}$
        \end{enumerate}
        In particular if $e^{\prime}=1,$ then $I(\tilde{y}) \subset I(\eta)=I(y),$ and moreover if $a \geq 1$ then
        $f: Y \rightarrow X$ is ramified along the irreducible component containing $y$.
    \end{proposition}
    \begin{block}{Proof}Omitted\end{block}
    \end{frame}\begin{frame}
    \begin{proposition}
        Let $f: Y \rightarrow X$ be a Galois covering of group $G$ where $Y$ and $X$ are semi-stable. Assume that $f_{K}: Y_{K} \rightarrow X_{K}$ is étale. Let $y$ be an ordinary double point of $Y,$ whose image in $X$ is a double point $x$. Let $C_{1}$ and $C_{2}$ be the two irreducible components of $Y_{k}$ passing through $y$ (which may be equal), and let $\eta_{1}$ and $\eta_{2}$ be the corresponding generic points of $Y_{k} .$ Then:

        \begin{enumerate}
            \item $I\left(\eta_{1}\right)$ and $I\left(\eta_{2}\right)$ are normal p-subgroups of $I(y),$ and they generate the (normal) p-sylow subgroup of $I(y)$
            \item the quotient $I(y) /\left\langle I\left(\eta_{1}\right), I\left(\eta_{2}\right)\right\rangle$ is a cyclic group of order prime to $p$
        \end{enumerate}
    \end{proposition}
    \end{frame}\begin{frame}
    \begin{block}{Proof}
        1. Etale localize to assume $y$ is the unique point of $Y$ above $x$ so that $G = I(y)$ and $C_1 \ne C_2$.
        Then $D(\eta_1) = D(\eta_2) = G$, and $I(\eta_1)$, $I(\eta_2)$ are $p$-groups.

        2. We can pass to the quotient curve $Y' = Y/\left\langle I\left(\eta_{1}\right), I\left(\eta_{2}\right)\right\rangle$ to trivialise this subgroup, then use local Riemann-Hurwitz. If $y'$ is the image of $y$ we have
        $$\mu _{y'} - 1 = n(\mu _x - 1) + d_K - d_k^w$$
        with $\mu _x=1$ and $d_K=0$ so $\mu _{y'}  = 0$. So the only possibility is $\mu _{y'} = 0$ and $d^w_k= 0$ so the cover $Y'_k \to X'_k$ is tamely ramified. So the original quotient $I(y) /\left\langle I\left(\eta_{1}\right), I\left(\eta_{2}\right)\right\rangle$ is cyclic.
    \end{block}
    \end{frame}\begin{frame}{A nicer cover}
        Let $X$ be a smooth proper $R$-curve with geom. conn. $X_K$ and $\{a_i : \Spec R \to X\}_{i = 1, \ldots r}$ all $R$-points of $X$ s.t. they have disjoint support (distinct on the special fibre).

        Let $f \colon  Y \to X$ be a galois cover with group $G$ s.t. $f_K \colon  Y_K \to X_K$ is etale away from the points $x_i$, $Y$ not necessarily smooth.

        After extending $R$ we can find $Y' \to Y$ proper birational with $Y'$ semi-stable and in such a way that the $G$-action extends (this follows from choosing  a minimal one).

        We can quotient to get $X' = Y'/G$.
    \end{frame}\begin{frame}{A nicer cover}
        The points $x_i$ induce points $x_i'$ on $X'$, which remain disjoint but may have support on a double point, to fix this blow up $X'$ and $Y'$ to get a semi-stable model $Y''$  with  a $G$-action and hence $X''=Y''/G$ in such a way that
        \begin{itemize}
            \item The irreducible components of the special fibre of $Y^{\prime \prime}$ are smooth.
            \item The integral points $\left\{x_{i}\right\}_{i=1}^{r}$ extend to points $\left\{x_{i}^{\prime \prime}\right\}_{i=1}^{r}$ of $X^{\prime \prime}(R)$ which have disjoint support and are contained in the smooth locus of $X^{\prime \prime}$
        \end{itemize}

        as $X$ was smooth originally and $X''$ is a semi-stable model of $X_K$ the special fibre of $X''$ is a tree with a bunch of $\PP^1$'s added to the original special fibre at double points.

\end{frame}






\begin{frame}{Back to Abhyankar}
    If $G$ is a quasi-$p$-group then it is generated by a family $(\alpha _1, \ldots, \alpha _m)$ of elements of $p$-power order, by adding new generators if needed we can assume $\alpha _1 \cdots \alpha _m = 1$.


    Consider a complete DVR $R$ with algebraically closed residue field $k$ of characteristic $p$ and fraction field $K$ of characteristic $0$, $\pi $ a uniformizer.

    Choose $m$ distinct $R$-points $x_1, \ldots, x_m$ of the projective line $\PP^1_R$, which have disjoint support (i.e.\ do not reduce to the same point on the special fibre).
    Write $$U = \PP^1_R\smallsetminus \{x_1, \ldots, x_m\}.$$

\end{frame}

\begin{frame}
    We consider the fundamental group of $U_{\overline K}$, the geometric generic fibre, as $K$ was assumed to be of characteristic $0$ the Lefschetz principle tells us that the answer agrees with the usual topological one (after profinite completion).

    The fundamental group is a free profinite group where we have $m$ topological generators  $(\sigma _1, \ldots, \sigma _m)$ satisfying $\sigma _1 \cdots \sigma _m = 1$, so that $G$ is a quotient of $\pi _1$. And we have a connected Galois cover $Y_{\overline K} \to \PP^1_{\overline K} $ with group $G$ etale away from $\{x_i\}_{i}$.

    The inertia subgroups above these points are cyclic $p$-groups.
    We can choose $\sigma _i$ to generate inertia above $x_i$.

    Enlarging $K$ everything is defined over $K$ and we can take integral versions of everything.
\end{frame}

\begin{frame}
    We can now apply the theory we developed to obtain semi-stable models $Y''$ and $X''$ for this setup where
    \begin{itemize}
        \item The irreducible components of the special fibre of $Y^{\prime \prime}$ are smooth.
        \item The rational points $\left\{x_{i}\right\}_{i=1}^{m}$ extend to integral points $\left\{x_{i}^{\prime \prime}\right\}_{i=1}^{m}$ of $X^{\prime \prime}(R)$ which have disjoint support and are contained in the smooth locus of $X^{\prime \prime}$
    \end{itemize}

    and where the special fibre of $X''$ is a tree of projective lines.
\end{frame}



\begin{frame}{Proving 3)}
    Assume we are in the case of 3); $G(S) \ne G$ and $G$ does not contain a non-trivial normal $p$-subgroup, we will apply the theory of semi-stable curves to prove that $G$ is rev-$p$.

    Combinatorial step shows that: the graph of the special fibre of $Y''$ with action of $G$ is a \terminology{graph with inertia} (satisfies 8 conditions see next week).

    It then says that there is a vertex $s$ of the graph of $Y''_k$ for which the decomposition group for that component is all of $G$, and whose image in the quotient tree is a leaf.
    As we have no non-trivial normal $p$-subgroup by assumption we have $I_s = 1$.

    Restricting to this component covering a single $\PP^1$ in the tree below as the vertex is a leaf it meets the rest of the special fibre only once, we have only one bad point in the component?, which when removed gives us an etale $G$-galois cover.
\end{frame}
\end{document}
