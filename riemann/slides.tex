\documentclass{beamer}

\usetheme{Frankfurt}
\usecolortheme{dolphin}

\usepackage{amsmath, amssymb, amsfonts}
\usepackage[utf8]{inputenc}
\usepackage[T1]{fontenc}
\usepackage[english]{babel}

\author{Alex J. Best}
\institute{WMS Talks}
\date{4/2/2014}
\title{Riemann Hypotheses}

\begin{document}

\frame{\titlepage}

\begin{frame}
\frametitle{In this talk:}
\tableofcontents
\end{frame}

\section{The original hypothesis}
\begin{frame}{The Riemann zeta function}
\begin{block}{A brief history:}
\begin{itemize}
\pause \item In ???? Euler found that
\[\sum_{n=1}^{\infty} \frac{1}{n^2} = \frac{\pi^2}{6}.\]
\pause \item He also discovered formulae for $\sum_{n=1}^{\infty} n^{-2k}$ in terms of the Bernoulli numbers $B_{2k}$ for all natural $k$.

\pause \item odd still unkown today?

\end{itemize}
\end{block}
\end{frame}

\begin{frame}{The Riemann zeta function}
\begin{block}{A brief history:}
\end{block}
\end{frame}

%plots

\begin{frame}{Ramanujan graphs}
\begin{block}{}
\end{block}
\end{frame}

\begin{frame}{The Ihara zeta function}
\begin{block}{}
\end{block}
\end{frame}

\begin{frame}{The zeta function of a scheme}
\begin{block}{}
\end{block}
\end{frame}

\section{The Dedekind zeta function}
\begin{frame}{The Dedekind zeta function}
Dedekind wanted to use the 
\begin{block}{}
\end{block}
\end{frame}
\note{didn't want to talk about number theory so much as I wanted to focus more on why you might care about zeta functions even if you don't care about the structure of the primes.}

\end{document}
