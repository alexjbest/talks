%\documentclass[notheorems]{beamer}
%\documentclass[handout]{beamer}
%\documentclass[handout,notes=show]{beamer}

%\usepackage[draft]{pdfcomment}

\usetheme{metropolis}
%\usecolortheme{dolphin}
% No navigation bars
\beamertemplatenavigationsymbolsempty

\usepackage{amsmath, amssymb, amsfonts, tikz}
\usepackage[utf8]{inputenc}
\usepackage[T1]{fontenc}
\usepackage[english]{babel}

\usepackage{amsthm}
\theoremstyle{plain}
\newtheorem{theorem}{Theorem}[section]
\newtheorem{corollary}[theorem]{Corollary}
\newtheorem{lemma}[theorem]{Lemma}
\newtheorem{algorithm}[theorem]{Algorithm}
\newtheorem{proposition}[theorem]{Proposition}
\newtheorem{claim}[theorem]{Claim}
\newtheorem{fact}[theorem]{Fact}
\newtheorem{conjecture}[theorem]{Conjecture}
%% Definition-like environments, normal text
%% Numbering is in sync with theorems, etc
\theoremstyle{definition}
\newtheorem{definition}[theorem]{Definition}
%% Remark-like environments, normal text
%% Numbering is in sync with theorems, etc
\theoremstyle{definition}
\newtheorem{remark}[theorem]{Remark}
\newtheorem{observation}[theorem]{Observation}
%% Example-like environments, normal text
%% Numbering is in sync with theorems, etc
\theoremstyle{definition}
\newtheorem{example}[theorem]{Example}
\newtheorem{question}[theorem]{Question}
\newcommand{\terminology}[1]{\textbf{#1}}

\newcommand{\pdfnote}[1]{\marginnote{\pdfcomment[icon=note]{#1}}}

\newcommand{\NN}{\mathbf{N}}
\newcommand{\ZZ}{\mathbf{Z}}
\newcommand{\QQ}{\mathbf Q}
\newcommand{\CC}{\mathbf C}
\newcommand{\RR}{\mathbf R}
\newcommand{\FF}{\mathbf F}
\newcommand{\dR}{\mathrm{dR}}
\newcommand{\lt}{<}
\newcommand{\gt}{>}
\newcommand{\amp}{&}
\newcommand{\diff}{\mathop{}\!\mathrm{d}}
\newcommand{\ints}{\mathcal{O}}
\newcommand{\ideal}[1]{\mathfrak{#1}}
\usepackage{mathrsfs}\usepackage{cancel}
\newcommand{\Gal}[2]{\operatorname{Gal}(#1/#2)}
\newcommand{\absgal}[1]{\operatorname{Gal}(\overline{#1}/#1)}
\newcommand{\units}{^{\times}}

\newcommand{\sheaf}[1]{\operatorname{\mathcal{#1}}}
\newcommand{\inv}{^{-1}}
\DeclareMathOperator{\norm}{Nm}
\DeclareMathOperator{\ord}{ord}
\DeclareMathOperator{\divisor}{div}
\DeclareMathOperator{\PP}{\mathbf{P}}
\DeclareMathOperator{\Hom}{Hom}
\DeclareMathOperator{\coker}{coker}

\newcommand{\lb}{[}
\newcommand{\rb}{]}

\author{Alex J. Best}
\institute{STAGE}
\date{18/3/2020}
\title{Motivation for $p$-adic modular forms}
\setbeamertemplate{footline}[text line]
{\url{https://alexjbest.github.io/talks/motivation-for-p-adic-modular-forms/slides_h.pdf}}
\begin{document}

\begin{frame}
  \titlepage

  \pdfnote[item]{Thank the audience for attending remotely in times of Covid.}
\end{frame}

%\begin{frame}
%\frametitle{Table of Contents}
%\tableofcontents[currentsection]
%\end{frame}

\begin{frame}{Overview}
    These slides are available online (in handout form) at

    \begin{centering}
        \url{https://alexjbest.github.io/talks/motivation-for-p-adic-modular-forms/slides_h.pdf}
    \end{centering} \pdfnote{to read along check back definitions, link will be at the bottom of all slides, pls no skip}\pause

    \textbf{Goal:} Introduce, post hoc, motivation for Katz's definition of $p$-adic modular forms, especially to motivate Serre's $\partial$ operator.
    \pdfnote{Abstract: In this talk we will introduce }

\end{frame}

\begin{frame}{Recall}
    A \emph{modular form} of weight $k$ is a function
    \[f\colon \{(E \xrightarrow \pi R\text{ an ell.\ curve}, \omega\in \Gamma(E, \Omega^1_{E/R})\text{ nowhere vanishing})\}\to R\] \pause
    s.t.
    \begin{enumerate}
        \item \(\forall \lambda\in R\units,  f(E, \lambda \omega) = \lambda ^{-k} f(E, \omega)\)g
        \item \(f(E, \omega)\) is isomorphism invariant.
        \item \(f\) is functorial w.r.t. $R$.
    \end{enumerate} \pause
    we then have
    \[f(E, \omega) \cdot \omega^{\otimes k} \in \Gamma(R, {\underbrace{\pi_*(\Omega^1_{E/R})}_{\underline \omega_{E/R}}}^{\otimes k})\]

    \pdfnote{.}
\end{frame}

\begin{frame}{De Rham cohomology}
    The sheaf of values $\underline \omega_{E/R}$ is a subsheaf of the de Rham cohomology of $E/R$:
    \[H_{\dR}^1(E/R) := \mathbb H^1(E, \Omega^\bullet_{E/R}) \]
    \[0\to \underline \omega_{E/R} \to H^1_\dR(E/R) \to \underbrace{ H^1 (E, \sheaf O_E)}_{=\underline \omega_{E/R}} \to 0\]\pause
    assuming $1/6\in R$ we can canonically split this sequence:\pause

    Fixing $(E, \omega)/R$ we have a unique pair of meromorphic functions with poles only at $\infty$, of orders $2$ and $3$ resp., denoted by $X,Y$ so that
    \[ \omega = \frac{\diff X }{Y}\text{ and } E \colon Y^2 = 4X^3  - g_2X - g_3,\,g_i\in R\]
    \pdfnote{.}
\end{frame}

\begin{frame}
    Then we have an inclusion of 2-term complexes
    \[
        (\sheaf O_E \to \Omega_{E/R}^1) \subseteq (\sheaf O_{E}(\infty) \to  \Omega_{E/R}^1 (2\infty))\text,
    \]\pause
    this induces an isomorphism on $\mathbb H^1$.
    Moreover for $i\gt0$,
    \[ H^i(E, \sheaf O_E(\infty)) = 0 \]
    \[ H^i(E, \Omega^1_{E/R}(2\infty)) = 0 \]
    giving
    \begin{align*}
        H^1_\dR(E/R) \amp\cong \mathbb H^1(E, \sheaf O_E(\infty) \to  \Omega^1_{E/R}(2\infty))\\
          \amp= \coker( H^0(E, \sheaf O_E(\infty)) \to H^0(E, \Omega^1_{E/R}(2\infty)))\\
                                                                        \amp= \coker( R \xrightarrow 0  H^0(E, \Omega^1_{E/R}(2\infty)))        \\
                                                                        \amp= H^0(E, \Omega^1_{E/R}(2\infty))
    \end{align*}
    \pdfnote{.}
\end{frame}

\begin{frame}
    \pdfnote{.}
\end{frame}

\begin{frame}
    \pdfnote{.}
\end{frame}

\begin{frame}

\begin{theorem}[Katz, Manin-Vishik]
\end{theorem}
    \pdfnote{.}

\end{frame}


\end{document}
