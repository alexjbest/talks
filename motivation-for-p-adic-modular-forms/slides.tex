%\documentclass[notheorems]{beamer}
%\documentclass[handout]{beamer}
%\documentclass[handout,notes=show]{beamer}

%\usepackage[draft]{pdfcomment}

\usetheme{metropolis}
%\usecolortheme{dolphin}
% No navigation bars
\beamertemplatenavigationsymbolsempty

\usepackage{amsmath, amssymb, amsfonts, tikz}
\usepackage[utf8]{inputenc}
\usepackage[T1]{fontenc}
\usepackage[english]{babel}
\usepackage{bm}


\usepackage{amsthm}
\theoremstyle{plain}
\newtheorem{theorem}{Theorem}[section]
\newtheorem{corollary}[theorem]{Corollary}
\newtheorem{lemma}[theorem]{Lemma}
\newtheorem{algorithm}[theorem]{Algorithm}
\newtheorem{proposition}[theorem]{Proposition}
\newtheorem{claim}[theorem]{Claim}
\newtheorem{fact}[theorem]{Fact}
\newtheorem{conjecture}[theorem]{Conjecture}
%% Definition-like environments, normal text
%% Numbering is in sync with theorems, etc
\theoremstyle{definition}
\newtheorem{definition}[theorem]{Definition}
%% Remark-like environments, normal text
%% Numbering is in sync with theorems, etc
\theoremstyle{definition}
\newtheorem{remark}[theorem]{Remark}
\newtheorem{observation}[theorem]{Observation}
%% Example-like environments, normal text
%% Numbering is in sync with theorems, etc
\theoremstyle{definition}
\newtheorem{example}[theorem]{Example}
\newtheorem{question}[theorem]{Question}
\newcommand{\terminology}[1]{\textbf{#1}}

\newcommand{\pdfnote}[1]{\marginnote{\pdfcomment[icon=note]{#1}}}

\newcommand{\NN}{\bm{N}}
\newcommand{\ZZ}{\bm{Z}}
\newcommand{\QQ}{\bm Q}
\newcommand{\CC}{\bm C}
\newcommand{\RR}{\bm R}
\newcommand{\FF}{\bm F}
\newcommand{\dR}{\mathrm{dR}}
\newcommand{\lt}{<}
\newcommand{\gt}{>}
\newcommand{\amp}{&}
\newcommand{\diff}{\mathop{}\!\mathrm{d}}
\newcommand{\ints}{\mathcal{O}}
\newcommand{\ideal}[1]{\mathfrak{#1}}
\usepackage{mathrsfs}\usepackage{cancel}
\newcommand{\Gal}[2]{\operatorname{Gal}(#1/#2)}
\newcommand{\absgal}[1]{\operatorname{Gal}(\overline{#1}/#1)}
\newcommand{\units}{^{\times}}

\newcommand{\sheaf}[1]{\operatorname{\mathcal{#1}}}
\newcommand{\inv}{^{-1}}
\DeclareMathOperator{\norm}{Nm}
\DeclareMathOperator{\ord}{ord}
\DeclareMathOperator{\divisor}{div}
\DeclareMathOperator{\PP}{\bm{P}}
\DeclareMathOperator{\Hom}{Hom}
\DeclareMathOperator{\coker}{coker}
\newcommand{\pair}[2]{\left\langle #1, #2 \right\rangle}

\newcommand{\lb}{[}
\newcommand{\rb}{]}

\author{Alex J. Best}
\institute{STAGE}
\date{18/3/2020}
\title{Motivation for $p$-adic modular forms}
\setbeamertemplate{footline}[text line]
{\url{https://alexjbest.github.io/talks/motivation-for-p-adic-modular-forms/slides_h.pdf}}
\begin{document}

\begin{frame}
    \titlepage

    \pdfnote[item]{Thank the audience for attending remotely in times of Covid.}
\end{frame}

%\begin{frame}
%\frametitle{Table of Contents}
%\tableofcontents[currentsection]
%\end{frame}

\begin{frame}{Overview}
    These slides are available online (in handout form) at

    \begin{centering}
        \url{https://alexjbest.github.io/talks/motivation-for-p-adic-modular-forms/slides_h.pdf}
    \end{centering} \pdfnote{to read along check back definitions, link will be at the bottom of all slides, pls no skip}\pause

    \textbf{Goal:} Introduce, post hoc, motivation for Katz's definition of $p$-adic modular forms, especially to motivate Serre's $\partial$ operator.
    \pdfnote{Abstract: In this talk we will introduce }

\end{frame}

\begin{frame}{Recall}
    A \emph{modular form} of weight $k$ is a function
    \[f\colon \{(E \xrightarrow \pi R\text{ an ell.\ curve}, \omega\in \Gamma(E, \Omega^1_{E/R})\text{ nowhere vanishing})\}\to R\] \pause
    s.t.
    \begin{enumerate}
        \item \(\forall \lambda\in R\units,  f(E, \lambda \omega) = \lambda ^{-k} f(E, \omega)\)
        \item \(f(E, \omega)\) is isomorphism invariant.
        \item \(f\) is functorial w.r.t. $R$.
    \end{enumerate} \pause
    we then have
    \[f(E, \omega) \cdot \omega^{\otimes k} \in \Gamma(R, {\underbrace{\pi_*(\Omega^1_{E/R})}_{\underline \omega_{E/R}}}^{\otimes k})\]

    \pdfnote{.}
\end{frame}

\begin{frame}{De Rham cohomology}
    The sheaf of values $\underline \omega_{E/R}$ is a subsheaf of the de Rham cohomology of $E/R$:
    \[0\to \underline \omega_{E/R} \to \overbrace{H^1_\dR(E/R)}^{ := \mathbb H^1(E, \Omega^\bullet_{E/R})} \to \underbrace{ H^1 (E, \sheaf O_E)}_{=\underline \omega_{E/R}^{\otimes-1}} \to 0\]\pause
    assuming $1/6\in R$ we can canonically split this sequence:

    Fixing $(E, \omega)/R$ we have a unique pair of meromorphic functions with poles only at $\infty$, of orders $2$ and $3$ resp., denoted by $X,Y$ so that
    \[ \omega = \frac{\diff X }{Y}\text{ and } E \colon Y^2 = 4X^3  - g_2X - g_3,\,g_i\in R\]
    \pdfnote{.}
\end{frame}

\begin{frame}
    Then we have an inclusion of 2-term complexes
    \[
        (\sheaf O_E \to \Omega_{E/R}^1) \subseteq (\sheaf O_{E}(\infty) \to  \Omega_{E/R}^1 (2\infty))\text,
    \]\pause
    this induces an isomorphism on $\mathbb H^1$.
    Moreover for $i\gt0$,
    \[ H^i(E, \sheaf O_E(\infty)) = 0 \]
    \[ H^i(E, \Omega^1_{E/R}(2\infty)) = 0 \]
    giving
    \begin{align*}
        H^1_\dR(E/R) \amp\cong \mathbb H^1(E, \sheaf O_E(\infty) \to  \Omega^1_{E/R}(2\infty))\\
        \amp= \coker( H^0(E, \sheaf O_E(\infty)) \to H^0(E, \Omega^1_{E/R}(2\infty)))\\
        \amp= \coker( R \xrightarrow 0  H^0(E, \Omega^1_{E/R}(2\infty)))        \\
        \amp= H^0(E, \Omega^1_{E/R}(2\infty))\\
        \amp \quad\ni \underbrace{\frac{\diff X(E, \omega)}{Y(E, \omega)}}_{=\omega},
        \underbrace{X(E, \omega)\cdot  \omega}_{=\eta}
    \end{align*}
    \pdfnote{.}
\end{frame}

\begin{frame}{How does $R\units$ act?}
    By uniqueness
    \begin{equation*}
        \begin{array}{l}
            {X}\left({E}, \lambda{\omega}\right)=\lambda^{-2} \cdot {X}({E}, \omega) \\
            {Y}({E}, \lambda \omega)=\lambda^{-3} \cdot {Y}({E}, \omega) \\
            {g}_{2}({E}, \lambda \omega)=\lambda^{-4} {g}_{2}({E}, \omega) \\
            {g}_{3}({E}, \lambda \omega)=\lambda^{-6} {g}_{3}({E}, \omega)
        \end{array}
    \end{equation*}\pause
    hence
    \begin{equation*}
        \lambda \omega=\frac{\diff {X}({E}, \lambda \omega)}{{Y}({E}, \lambda \omega)} \quad \text { and } \quad \lambda^{-1} \eta=\frac{{X}({E}, \lambda \omega) {\diff} {X}({E}, \lambda \omega)}{{Y}({E}, \lambda \omega)}
    \end{equation*}
    \begin{equation*}
        {H}_{\dR}^1{({E} / {R})} \simeq \underline{\omega}_{{E} / {R}} \oplus \underline{\omega}_{{E} / {R}}^{-1}
    \end{equation*}
    \begin{equation*}
        \text{Symm}^{{k}}\left({H}_{{\dR}}^{1}({E} / {R})\right) \simeq\left(\underline{\omega}_{{E} / {R}}\right)^{\otimes {k}} \oplus\left(\underline{\omega}_{{E} / {R}}\right)^{\otimes {k}-2} \oplus \cdots \oplus\left(\underline{\omega}_{{E} / {R}}\right)^{\otimes-{k}}
    \end{equation*}
    \pdfnote{.}
\end{frame}

\begin{frame}{Connections}
    Let $f: S \rightarrow T$ be a smooth $T$-scheme, $\mathcal{E}$ a quasi-coherent sheaf of $\mathcal{O}_{S}$ -modules.
    A \emph{connection} on $\mathcal{E}$ is a homomorphism
    $$
    \nabla: \mathcal{E} \rightarrow \mathcal{E} \otimes_{\mathcal{O}_{X}} \Omega_{S / T}^{1}
    $$
    of abelian sheaves satisfying the "Leibniz rule"
    $$
    \nabla(g e)=g \nabla(e)+e\otimes\diff g 
    $$
    where $g$ and $e$ are sections of $\mathcal{O}_{S}$ and $\mathcal{E},$ respectively, over an open subset of $S$ and $\diff : \mathcal{O}_{S} \rightarrow \Omega_{S / T}^{1}$ the exterior derivative.\pause

    Given an element of the tangent bundle $t \in (\Omega_{S/T}^1)^*$ we can define
    \[
        \nabla_t \colon \mathcal E \to \mathcal{E} \otimes_{\mathcal{O}_{X}} \Omega_{S / T}^{1}\to \mathcal E
    \]
    by ``contraction''.


\end{frame}

\begin{frame}{The Gauss-Manin connection -- complex case}
    Let $R$ be the ring of holomorphic functions of $\tau$, and $E$ be the relative elliptic curve
    \[\CC/ \ZZ + \ZZ \tau\]
    which can be expressed as 
    $y^{2}=4 x^{3}-\frac{E_4}{12} x+\frac{E_{6}}{216},\,E_i \in R$.\pause

    The de Rham homology $H^\dR_1(E/R)$ is then free on $\gamma_1,\gamma_2$, the paths $\overset{\to}{0\tau},\overset{\to}{01}$ respectively.\pause

    The Gauss-Manin connection is a connection on $H^1_\dR(E/R)$, defined as a differential in a spectral sequence coming from taking $p$-forms with only $q$ terms from $E$.\pause

    To give the Gauss-Manin connection in this context we need only define
    \(\nabla_\tau = \nabla_{\diff/\diff \tau}\), we do so via the dual connection on $H_1^\dR(E/R)$ as
    \begin{equation*}
        \begin{array}{l}
            \int_{\gamma_{{i}}} \nabla_{\tau}(\xi)=\frac{{\diff}}{{\diff} \tau} \int_{\gamma_{i}} \xi \quad \text { for } \quad \xi \in {H}_{\dR}^{1}({E} / {R}), \text { and } i=1,2
        \end{array}
    \end{equation*}
    \pdfnote{.}
\end{frame}

\begin{frame}{Computation of the connection matrix}
    Let
    \[ \omega = \frac{\diff x}{y},\,\eta = \frac{x \diff x}{y}\]
    Poincaré duality gives elements $\gamma_i \in H^1_\dR(E/R)$ also, satisfying
    \[\pair{\gamma_2}{\gamma_1} = 1 = -\pair{\gamma_1}{\gamma_2} \]
    \[\pair{\gamma_1}{\gamma_1} = 0 = \pair{\gamma_2}{\gamma_2} \]
    we then define
    \[ \omega_i  =\int_{\gamma_i} \omega = \pair{\omega}{\gamma_i},\,\eta_i = \int_{\gamma_i} \eta = \pair{\eta}{\gamma_i}\in R\]
    so that we have
    $$
    \left(\begin{array}{cc}
            \omega_{1} & -\omega_{2} \\
            \eta_{1} & -\eta_{2}
            \end{array}\right)\left(\begin{array}{l}
            \gamma_{2} \\
            \gamma_{1}
            \end{array}\right)=\left(\begin{array}{l}
            \omega \\
            \eta
    \end{array}\right)
    $$
    \pdfnote{.}
\end{frame}

\begin{frame}
    $$
    \left(\begin{array}{cc}
            \omega_{1} & -\omega_{2} \\
            \eta_{1} & -\eta_{2}
            \end{array}\right)\left(\begin{array}{l}
            \gamma_{2} \\
            \gamma_{1}
            \end{array}\right)=\left(\begin{array}{l}
            \omega \\
            \eta
    \end{array}\right)
    $$
    to invert this we note that
    $$
    \eta_{1} \omega_{2}-\eta_{2} \omega_{1}=2 \pi i
    $$
    so
    $$
    2 \pi i\left(\begin{array}{l}
            \gamma_{2} \\
            \gamma_{1}
            \end{array}\right)=\left(\begin{array}{ll}
            -\eta_{2} & \omega_{2} \\
            -\eta_{1} & \omega_{1}
            \end{array}\right)\left(\begin{array}{l}
            \omega \\
            \eta
    \end{array}\right)
    $$
    \pdfnote{Legendre's period relations}
    to which we want to apply $\nabla_\tau$
    $$
    \int_{{\gamma}_{{i}}} \nabla_{{\tau}}(\gamma_j)=\frac{\mathrm{d}}{\mathrm{d} \tau} \int_{\gamma_{{i}}}\gamma_j = 0
    $$
\end{frame}

\begin{frame}
    \begin{align*}
        \left(\begin{array}{l}
                0\\
                0
                \end{array}\right)\amp=\nabla_\tau\left(\left(\begin{array}{ll}
                    -\eta_{2} & \omega_{2} \\
                    -\eta_{1} & \omega_{1}
                    \end{array}\right)\left(\begin{array}{l}
                    \omega \\
                    \eta
        \end{array}\right)\right)\\\amp=
        \left(\begin{array}{cc}
                -\frac{\diff}{\diff\tau}\eta_{2}^{} & \frac{\diff}{\diff\tau}\omega_{2}^{} \\
                -\frac{\diff}{\diff\tau} \eta_{1}^{} & \frac{\diff}{\diff\tau}\omega_{1}^{}
                \end{array}\right)\left(\begin{array}{l}
                \omega \\
                \eta
                \end{array}\right)+\left(\begin{array}{ll}
                -\eta_{2} & \omega_{2} \\
                -\eta_{1} & \omega_{1}
                \end{array}\right)\left(\begin{array}{c}
                \nabla_{\tau}(\omega) \\
                \nabla_{\tau}(\eta)
        \end{array}\right)
    \end{align*}\pause
    so we get
    \begin{align*}
        \left(\begin{array}{c}
                \nabla_{\tau}(\omega) \\
                \nabla_{\tau}(\eta)
                \end{array}\right) &=\frac{-1}{2 \pi i}\left(\begin{array}{cc}
                \omega_{1} & -\omega_{2} \\
                \eta_{1} & -\eta_{2}
                \end{array}\right)\left(\begin{array}{cc}
                -\eta_{2}^{\prime} & \omega_{2}^{\prime} \\
                -\eta_{1}^{\prime} & \omega_{1}^{\prime}
                \end{array}\right)\left(\begin{array}{l}
                \omega \\
                \eta
        \end{array}\right) \\
&=\frac{-1}{2 \pi i}\left(\begin{array}{cc}
        \eta_{1}^{\prime} \omega_{2}-\eta_{2}^{\prime} \omega_{1} & \omega_{1} \omega_{2}^{\prime}-\omega_{2} \omega_{1}^{\prime} \\
        \eta_{2} \eta_{1}^{\prime}-\eta_{1} \eta_{2}^{\prime} & \eta_{1} \omega_{2}^{\prime}-\eta_{2} \omega_{1}^{\prime}
        \end{array}\right)\left(\begin{array}{c}
        \omega \\
        \eta
\end{array}\right)
    \end{align*}

    \pdfnote{.}
\end{frame}

\begin{frame}
    In fact $\omega_1= \tau$ and $\omega_2 = 1$ so that $\omega_1' = 1, \omega_2' = 0$ and $\eta_1 - \tau \eta_2 = 2 \pi i$, giving $\eta_1' -\tau \eta_2' = \eta_2$ and we simplify 
    \begin{align*}
        \left(\begin{array}{c}
                \nabla_{\tau}(\omega) \\
                \nabla_{\tau}(\eta)
        \end{array}\right)
        \amp=\frac{-1}{2 \pi i}\left(\begin{array}{cc}
                \eta_{1}^{\prime} \omega_{2}-\eta_{2}^{\prime} \omega_{1} & \omega_{1} \omega_{2}^{\prime}-\omega_{2} \omega_{1}^{\prime} \\
                \eta_{2} \eta_{1}^{\prime}-\eta_{1} \eta_{2}^{\prime} & \eta_{1} \omega_{2}^{\prime}-\eta_{2} \omega_{1}^{\prime}
                \end{array}\right)\left(\begin{array}{c}
                \omega \\
                \eta
        \end{array}\right)\\
        \amp=\frac{-1}{2 \pi i}\left(\begin{array}{cc}
                \eta_{1}^{\prime}-\eta_{2}^{\prime} \tau & -1 \\
                \eta_{2} \eta_{1}^{\prime}-\eta_{1} \eta_{2}^{\prime} & -\eta_{2}
                \end{array}\right)\left(\begin{array}{c}
                \omega \\
                \eta
        \end{array}\right)\\
        \amp=\frac{-1}{2 \pi i}\left(\begin{array}{cc}
                \eta_{2} & -1 \\
                \left(\eta_{2}\right)^{2}-2 \pi i \eta_{2}^{\prime} & -\eta_{2}
                \end{array}\right)\left(\begin{array}{l}
                \omega \\
                \eta
        \end{array}\right)
    \end{align*}
    purely in terms of $\eta_2$.
    \pdfnote{.}
\end{frame}

\begin{frame}{Determining $\eta_2$}

    Let $q = e^{2\pi i \tau}$, and
    \[
        P(q)=E_2(q) =1-
        24 \sum_{n \geq 1} \sigma_{1}(n) q^{n}, \text { where } \sigma_{1}(n)=\sum_{d \geq 1, d | n} d
    \] \pdfnote{quasi modular not modular} then
    \begin{lemma}
        $$
        \eta_{2}=-\sum_{m} \sum_{n}\left.^{\prime}\right. \frac{1}{(m \tau+n)^{2}}=\frac{-\pi^{2}}{3} P
        $$
    \end{lemma}
    \pdfnote{the order of this sum matters!}
\end{frame}

\begin{frame}
    \begin{lemma}
        $$
    \eta_{2}=-\sum_{m} \sum_{n}{}^{\prime}\frac{1}{(m \tau+n)^{2}}=\frac{-\pi^{2}}{3} P
        $$
    \end{lemma}
    \begin{proof}
        \[ \eta=X \diff X / Y=\wp(z) \diff z = -\diff \zeta \]
    \only<2->{
        where
            \[
                \zeta = \frac{1}{z} + \sum_{m} \sum_{n}{}^{\prime}\left(\frac{1}{z-m \tau-n}+\frac{1}{m\tau+n}+\frac{z}{(m \tau+n)^{2}}\right)
            \]
        }
    \only<3->{
            \begin{align*}
            \eta_{2}\only<3-4>{\amp=\int_{\gamma_{2}} \eta=\int_{0}^{1}(-\diff  \zeta(z))=\int_{\mathbf{z}}^{\mathbf{z}+1}(-\diff  \zeta(\mathbf{z}))=\zeta(\mathbf{z})-\zeta(\mathbf{z}+1)\\
         &=\frac{1}{2}-\frac{1}{z+1}+\sum_{m} \sum_{n}{}^{\prime}\left\{\frac{1}{z-m \tau-n}-\frac{1}{z-m \tau-n+1}-\frac{1}{(m \tau+n)^{2}}\right\} \\}
         \only<5->{&=\frac{1}{z}-\frac{1}{z+1}+\sum_{m \neq 0} \sum_{n} \frac{-1}{(m \tau+n)^{2}}+\sum_{n \neq 0}\left\{\frac{-1}{n^{2}}+\frac{1}{z-n}-\frac{1}{z+1-n}\right\} \\}
         \only<6->{&=-\sum_{m} \sum_{n}{}^{\prime} \frac{1}{(m \tau+n)^{2}}}
        \end{align*}
    }
    \end{proof}
\end{frame}

\begin{frame}{Aside on $\eta_1$}
    Similarly
    \[
        \eta_{1}=\zeta(z)-\zeta(z+\tau)=-\sum_{n} \sum_{m}{}^{\prime} \frac{\tau}{(m \tau+n)^{2}}
    \]
    giving
    \[
        \eta_{2}(-1 / \tau)=\tau_{\eta_{1}}(\tau)
    \]
    so
    \[
        \begin{aligned}
&\frac{\eta_{2}(-1 / \tau)}{\tau}-\tau \eta_{2}(\tau)=2 \pi i\\
\implies&\eta_{2}(-1 / \tau)=\tau^{2} \eta_{2}(\tau)+2 \pi \mathbf{i} \tau
        \end{aligned}
    \]
    \[
        P(-1 / \tau)=\tau^{2} P(\tau)-\frac{6i \tau}{\pi}
    \]
    \pdfnote{.}
\end{frame}

\begin{frame}
    In conclusion we have
    \[
        \left(\begin{array}{c}
\nabla_\tau(\omega) \\
\nabla_{\tau}(\eta)
\end{array}\right)=\frac{1}{2 \pi i}\left(\begin{array}{cc}
\frac{\pi^{2} P}{3} & 1 \\
\frac{\pi^{4}}{9} P^{2}-\frac{12}{2 \pi i} P^{\prime} & -\frac{\pi^{2}}{3} P
\end{array}\right)\left(\begin{array}{l}
\omega \\
\eta
\end{array}\right)
\]
We can consider this in terms of $\omega_{\mathrm{can}}=2 \pi i \omega,\, \eta_{\mathrm{can}}=\frac{1}{2 \pi i} \eta$ and $\theta=\frac{1}{2 \pi i} \frac{\diff}{\diff \tau}=q \frac{\diff}{\diff q}$
For the Tate curve over $\CC((q))$, we have $\omega_{can} = \diff t / t$.
\[
\nabla(\theta)\left(\begin{array}{c}
\omega \\
\eta
\end{array}\right)=\left(\begin{array}{cc}
\frac{-P}{12} & \frac{-1}{4 \pi^{2}} \\
\frac{\pi^{2}}{36}\left(P ^{2}-12 \theta P \right) & \frac{P }{12}
\end{array}\right)\left(\begin{array}{c}
\omega \\
\eta
\end{array}\right)
\]
so
\[
    \nabla(\theta)\left(\begin{array}{c}
\omega_{\mathrm{can}} \\
\eta_{\mathrm{can}}
\end{array}\right)=\left(\begin{array}{cc}
\frac{-P}{12} & 1 \\
\frac{P^{2}-12 \theta {P}}{144} & \frac{{P}}{12}
\end{array}\right)\left(\begin{array}{l}
\omega_{\mathrm{can}} \\
\eta_{\mathrm{can}}
\end{array}\right)
\]
    \pdfnote{.}
\end{frame}

\begin{frame}{Kodaira-Spencer}
    As before let $f\colon S \rightarrow T$ be a smooth $T$-scheme, $E/S$ an elliptic curve.

    We can take a nowhere vanishing invariant differential $\omega \in \Omega_{E/S}^1$ and a derivation $D\in Der(S/T)$ and form the cup product
    \[ \pair{\omega}{\nabla(D) \omega} \in \sheaf O_S\]
    as we have $\omega$ in both sides this defines a pairing between $Der(S/T)$ and $\underline \omega^2$ for
    \[\underline \omega = \pi_{*} \Omega_{E / S}^{1} \]
    this gives a map
    \[
        \underline{\omega}^{\otimes 2}\to \Omega_{\mathrm{S} / \mathrm{T}}^{1}
    \]
    \pdfnote{.}
\end{frame}

\begin{frame}
    \begin{lemma}
        On the Tate curve over $\ZZ((q))$
        \[
            \omega_{can}^{\otimes 2} \mapsto \diff q / q
        \]
    \end{lemma}
    \begin{proof}
        We must check that
        \[\pair{\omega_{can}}{\nabla(\theta) \omega_{can}}_\dR = 1\]
        but we already found
        \[
            \nabla(\theta)\left(\omega_{\mathrm{can}}\right)=\frac{-P}{12} \omega_{\mathrm{can}}+\eta_{\mathrm{can}}
        \]
    \end{proof}
    \pdfnote{.}
\end{frame}
\begin{frame}{The Gauss-Manin connection}
    To determine if a $q$-expansion $f(q) \in \CC\lb\lb q\rb\rb$ is a modular form of weight $k$ we must check if
    \[f(q) (\omega_{can})^{\otimes k}\]
    extends to all of $\underline \omega^{\otimes k}$.
    Viewing this inside of $H^1_\dR$ we ask instead that there exist $a,b\in \NN$ with $a-b = k$ such that
    \[f(q) (\omega_{can})^{\otimes a}(\eta_{can})^{\otimes b}\]
    extends to 
    \[\text{Symm}^{{a+b}}\left({H}_{{\dR}}^{1}({E} / {S})\right) \]\pause
    The Gauss-Manin connection is
    \[
        \nabla: \mathrm{H}_{\mathrm{dR}}^{1}(\mathrm{E} / \mathrm{S}) \longrightarrow \mathrm{H}_{\mathrm{dR}}^{1}(\mathrm{E} / \mathrm{S}) \otimes \Omega_{\mathrm{S}/T}^{1} 
    \]
\end{frame}

\begin{frame}
    We can tensor to get
    \[
        \nabla \colon \text {Symm}^{\mathrm{k}}\left(\mathrm{H}^{1}\right) \longrightarrow \text{Symm}^{\mathrm{k}}\left(\mathrm{H}^{1}\right) \otimes \Omega_{S/T}^{\mathbf{1}}
    \]
    if
    \[
        \Omega_{S / T}^{1} \simeq \underline{\omega}^{\otimes 2}
    \]
    we can view this as
    \[
    \sum_{{j}=0}^{k} \underline{\omega}^{\otimes {k}-2 {j}} \longrightarrow \sum_{{j}=0}^{{k}} \underline{\omega}^{\otimes {k}-2 {j}} \otimes \omega^{\otimes 2}=\sum_{j=0}^{{k}} \underline{\omega}^{\otimes {k}+2-2 {j}}
    \]
    \pdfnote{.}
\end{frame}

\begin{frame}{The image of $f(q)$}
    Under the Gauss-Manin connection
    \begin{align*}
        f\mapsto
\amp\theta(f) \cdot\left(\omega_{can}\right)^{\otimes 2} \cdot\left(\omega_{can}\right)^{\otimes a} \cdot\left(\eta_{can}\right)^{\otimes b} \\
\amp+f \cdot a \cdot\left(\omega_{can}\right)^{\otimes a-1}\left(\frac{-P}{12} \omega_{can}+\eta_{can}\right) \otimes\left(\omega_{can}\right)^{\otimes 2} \otimes\left(\eta_{can}\right)^{b} \\
\amp+f \cdot\left(\omega_{can}\right)^{\otimes a} \cdot b \cdot\left(\eta_{can}\right)^{\otimes b-1}\left(\frac{P^{2}-12 \theta P}{144} \omega_{can}+\frac{P}{12} \eta_{can}\right) \cdot\left(\omega_{can}\right)^{\otimes 2}
    \end{align*}
    \begin{align*}
        =\amp \left( \theta(f) - (a-b)f \frac P{12}\right)\left(\omega_{can}\right)^{\otimes a+ 2}\left(\eta_{can}\right)^{\otimes b}\\
    \amp+ \left( af \right)\left(\omega_{can}\right)^{\otimes a+ 1}\left(\eta_{can}\right)^{\otimes b + 1}\\
    \amp+ \left( bf \frac{P^{2}-12 \theta P}{144} \right)\left(\omega_{can}\right)^{\otimes a+ 3}\left(\eta_{can}\right)^{\otimes b -1}
    \end{align*}

    \pdfnote{.}
\end{frame}

\begin{frame}
    So if $f$ is modular of weight $k = a-b$ then
    \[
    \begin{cases}
        \theta(f) - (a-b)f \frac P{12},\amp \text{ is modular of weight } a+ 2 - b\\
     af,\amp \text{ is modular of weight } a - b \\
    bf \frac{P^{2}-12 \theta P}{144},\amp \text{ is modular of weight } a + 3 - b + 1
\end{cases}
    \] 
    the operator
    \[ \partial (F ) = 12 \theta (f ) - k P f\]
    due to Serre therefore raises the weight by 2.
    
\end{frame}

\begin{frame}{$E_4$}
    \begin{corollary}
        \[P^2 - 12 \theta P\]
        is modular of weight 4 and hence
        \[P^2 - 12 \theta P = E_4 = Q\]
    \end{corollary}
    \begin{proof}
        Start with $f=1$ modular of weight $1 -1 = 0$ to see $P^2 - 12 \theta P$ has weight 4.
    \end{proof}
\end{frame}

\begin{frame}{$P$}
    \begin{corollary}[Deligne]
        \[P =  \frac{\theta \Delta}{\Delta}\]
    \end{corollary}
    \begin{proof}
        \[ \theta(\Delta) -  \Delta P =  0\]
        as it lies in weight 14 and level 1.

    \end{proof}
\end{frame}


\end{document}
