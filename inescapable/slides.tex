\documentclass[notheorems]{beamer}
%\documentclass[handout]{beamer}
%\documentclass[handout,notes=show]{beamer}

\usetheme{metropolis}
%\usecolortheme{dolphin}
% No navigation bars
\beamertemplatenavigationsymbolsempty

\usepackage{amsmath, amssymb, amsfonts, tikz}
\usepackage[utf8]{inputenc}
\usepackage[T1]{fontenc}
\usepackage[english]{babel}

\usepackage{amsthm}
\theoremstyle{plain}
\newtheorem{theorem}{Theorem}[section]
\newtheorem{corollary}[theorem]{Corollary}
\newtheorem{lemma}[theorem]{Lemma}
\newtheorem{algorithm}[theorem]{Algorithm}
\newtheorem{proposition}[theorem]{Proposition}
\newtheorem{claim}[theorem]{Claim}
\newtheorem{fact}[theorem]{Fact}
\newtheorem{conjecture}[theorem]{Conjecture}
%% Definition-like environments, normal text
%% Numbering is in sync with theorems, etc
\theoremstyle{definition}
\newtheorem{definition}[theorem]{Definition}
%% Remark-like environments, normal text
%% Numbering is in sync with theorems, etc
\theoremstyle{definition}
\newtheorem{remark}[theorem]{Remark}
\newtheorem{observation}[theorem]{Observation}
%% Example-like environments, normal text
%% Numbering is in sync with theorems, etc
\theoremstyle{definition}
\newtheorem{example}[theorem]{Example}
\newtheorem{question}[theorem]{Question}
\newcommand{\terminology}[1]{\textbf{#1}}

\newcommand{\NN}{\mathbf{N}}
\newcommand{\ZZ}{\mathbf{Z}}
\newcommand{\QQ}{\mathbf Q}
\newcommand{\lt}{<}
\newcommand{\gt}{>}
\newcommand{\amp}{&}

\usepackage{enumitem}



\author{Alex J. Best}
\institute{BU Math Retreat 2018}
\date{5/5/2018}
\title{The (inescapable) $p$-adics}

\begin{document}

\maketitle

%\begin{frame}
%\frametitle{Table of Contents}
%\tableofcontents[currentsection]
%\end{frame}

\begin{frame}{Linear recurrence sequences}
\begin{definition}[Linear recurrence sequence]
A \terminology{linear recurrence sequence}, is a sequences whose \(n\)th term is the linear combination of the previous \(k\) terms (for all \(n \ge k\))
\end{definition}
\only<2-3>{\begin{example}[Fibonacci]\label{example-27}
\(a_0 = 0, a_1 = 1\) and \(a_{n} = a_{n-1} + a_{n-2}\) for \(n \ge k =2\):
\begin{equation*}
0, 1, 1, 2, 3, 5, 8, 13, 21, 34, 55, 89, 144, 233, 377, 610, 987, 1597, 2584, 4181, 6765, 10946, 17711, 28657, 46368, 75025, 121393, 196418, 317811, 514229, 832040, 1346269, 2178309, 3524578, 5702887, 9227465, 14930352, 24157817, 39088169, 63245986, 102334155, \ldots\text{.}
\end{equation*}
\only<3>{\(a_n\) grows exponentially.}
%
\end{example}}

\only<4-5>{\begin{example}[A periodic sequence]\label{example-28}
\(a_0 = 1, a_1 = 0\) with \(a_n = -a_{n-1} - a_{n-2}\)
\begin{equation*}
1,0,-1,1,0,-1,1,0,-1,1,0,-1,1,0,-1,1,0,-1,1,0,-1,1,0,-1,1,0,-1,\ldots\text{.}
\end{equation*}
\only<5>{\(a_n\) is periodic now.}
\end{example}
}
\only<6-7>{\begin{example}[Natural numbers interlaced with zeroes]\label{example-29}
\(a_0= 1,a_1=0,a_2 = 2,a_3= 0\) with \(a_n = 2a_{n-2} - a_{n-4}\)%
\begin{equation*}
	1,0,2,0,3,0,4,0,5,0,6,0,7,0,8,0,9,0,10,0,11,0,12,0,13,0,14,0,15,0,16,0,17,0,\ldots
\end{equation*}
\only<7>{not periodic but the zeroes \emph{do} have a regular repeating pattern.}
\end{example}}
\end{frame}

\begin{frame}{The ultimate question}
\begin{question}
What possible patterns are there for the zeroes of a linear recurrence sequence?%
\end{question}
\pause
\begin{observation}
A linear recurrence sequence is the Taylor expansion around 0 of a rational function
\begin{equation*}
\frac{a_1  + a_2 x+ \cdots + a_\ell x^\ell}{b_1 + b_2 x \cdots + b_k x^k}
\end{equation*}
with \(b_1 \ne 0\) (so that the expansion makes sense).%
\end{observation}
\end{frame}

\begin{frame}{Linear recurrence sequences}

\begin{example}
\begin{equation*}
\frac{x}{1 - x - x^2}\text{.}
\leftrightarrow
\text{Fibonacci}
\end{equation*} \pause
\begin{equation*}
\frac{1}{1 + x + x^2}\text{.}
\leftrightarrow
1,0,-1,1,0,-1,1,0,-1,1,0,-1,1,0,-1,1,0,-1,1,0,-1,1,0,-1,1,0,-1,1,0,-1,1,0,-1,1,0,-1,1,0,-1,1,0,-1,\ldots
\end{equation*} \pause
\begin{equation*}
\frac{1}{(1-x^2)^2}\text{.}
\leftrightarrow
1,0,2,0,3,0,4,0,5,0,6,0,7,0,8,0,9,0,10,0,11,0,12,0,13,0,14,0,\ldots
\end{equation*} \pause
\begin{align*}
\frac{(1+x)^3-x^3}{(1+x)^5-x^5}
\leftrightarrow
& 1, -2, 3, -5, 10, -20, 35, -50, 50, 0, -175, 625,\\
& -1625, 3625, -7250, 13125, -21250, 29375, -29375, \\
& 0, 106250, -384375, 1006250, -2250000, 4500000, \\
& -8140625, 13171875, -18203125, 18203125, 0, -65859375, 238281250, -623828125, 1394921875, -2789843750, 5046875000, -8166015625,\ldots
\end{align*}
\end{example}
\end{frame}

\begin{frame}{Consequences}
\begin{observation}
The set of all linear recurrence sequences is a vector space! Hard to tell how the rule changes.
\end{observation}
\pause
We can always mess up a finite amount of behaviour. So assume \(a_n\) has infinitely many zeroes, what is the structure of the zero set?%
\end{frame}

\begin{frame}{Linear recurrence sequences}
\begin{example}
\begin{equation*}
\frac{1}{(1-x^2)^2} - (1 - x + 2x^2 + 3x^4 + 4x^6)
\leftrightarrow 0,1,0,0,0,0,0,0,5,0,6,0,7,0,8,0,\ldots\text{.}
\end{equation*}
\pause
\end{example}
\emph{Interlacing with 0} and \emph{shifting} correspond to plugging in \(x^2\) and multiplying by \(x\) respectively in the rational functions
\pause
\begin{equation*}
	\frac{1}{(1-x)^2} \leftrightarrow 1,2,3,4,5,6,7,8,9,10,11,12,13,14,15,16,17,18,19,20,21,22,23,24,25,26,26,\ldots
\end{equation*}
\pause
%
\begin{equation*}
	\frac{1}{(1-x^2)^2} \leftrightarrow 1,0,2,0,3,0,4,0,5,0,6,0,7,0,8,0,9,0,10,0,11,0,12,0,13,0,14,0,\ldots
\end{equation*}
\end{frame}

\begin{frame}{Linear recurrence sequences}
\begin{equation*}
	\frac{1}{(1-x)^2} \leftrightarrow 1,2,3,4,5,6,7,8,9,10,11,12,13,14,15,16,17,18,19,20,21,22,23,24,25,26,26,\ldots
\end{equation*}
\begin{equation*}
	\frac{1}{(1-x^2)^2} \leftrightarrow 1,0,2,0,3,0,4,0,5,0,6,0,7,0,8,0,9,0,10,0,11,0,12,0,13,0,14,0,\ldots
\end{equation*}
\begin{equation*}
\frac{1}{(1-x^4)^2} \leftrightarrow 1,0,0,0,2,0,0,0,3,0,0,0,4,0,0,0,5,0,0,0,6,0,0,0,7,0,0,0,\ldots
\end{equation*}
\pause
%
\begin{equation*}
\frac{x}{(1-x^4)^2} \leftrightarrow 0,1,0,0,0,2,0,0,0,3,0,0,0,4,0,0,0,5,0,0,0,6,0,0,0,7,0,0,0,\ldots
\end{equation*}
\pause
%
\begin{equation*}
\frac{1+2x}{(1-x^4)^2} \leftrightarrow 1,2,0,0,2,4,0,0,3,6,0,0,4,8,0,0,5,10,0,0,6,12,0,0,7,14,0,0,\ldots
\end{equation*}
\pause
Still has periodic zero set, all \(n\) congruent to \(2,3\) modulo 4.%
\par
\end{frame}

\begin{frame}{Approach}
Expand into partial fractions%
\begin{equation*}
\frac{p(x)}{q(x)} = \sum_{i = 1}^m \sum_{j=1}^{n_j} \frac{r_{ij}}{(1-\alpha_i x)^j}
\end{equation*}
\pause
do some math:
\begin{equation*}
\sum_{n=0}^\infty \left(\sum_{i = 1}^m \sum_{j=1}^{n_j} r_{ij} \binom{n+j-1}{j-1}  \alpha_i^n\right) x^n
\end{equation*}
\pause
Upshot: there are polynomials \(A_i(n)\) such that%
\begin{equation*}
a_n = \sum_{i=1}^m A_i(n)\alpha_i^n\text{.}
\end{equation*}
Like that formula for Fibonacci with the golden ratio in.
\end{frame}

\begin{frame}{Approach}

So \(a_n\) is an analytic function of \(n\) which has zeroes for infinitely many integer values.
\par\pause
Like \[\sin(\pi x)!\]
\par\pause
\begin{alertblock}{Ridiculous suggestion}
What if the integers were bounded? In that case infinitely many zeroes \(\implies\) the function is zero!
\end{alertblock}
\end{frame}

\begin{frame}
\begin{theorem}[{Ostrowski}]\label{theorem-36}
The only absolute values on \(\QQ\) are%
\begin{equation*}
	\text{the usual one}\,\, \&\,\, |\cdot|_p
\end{equation*}
defined by \(|p|_p = \frac1p\) and \(|q|_p = 1\) for all other primes \(q \ne p\).%
\end{theorem}
\pause
With \(|\cdot|_p\) the integers are bounded!
\pause
Are the functions%
\begin{equation*}
\sum_{i=1}^m A_i(n)\alpha_i^n
\end{equation*}
\(p\)-adic analytic functions of \(n\)?%
\par
\pause
\begin{alertblock}{Problem}
The \(p\)-adic exponential function has finite radius of convergence.
\end{alertblock}
\pause
\begin{exampleblock}{The fix}
\emph{Choose} \(p\) so that \(|\alpha_i|_p = 1\) for all \(i\), then \(\alpha_i^{p-1} = 1 + \lambda_i\) with \(|\lambda_i|_p \le \frac 1p\).
Now \((\alpha_i^{p-1})^n\) is analytic!%
\end{exampleblock}
\end{frame}

\begin{frame}
Write \(n\) as \(r + (p-1)n'\) with \(0\le r \lt p-1\)\pause, then
\begin{equation*}
a_n = \sum_{i=1}^m A_i(n)\alpha_i^n = \sum_{i=1}^m A_i(r + (p-1)n')\alpha_i^{r + (p-1)n'}
\end{equation*}
%
\begin{equation*}
= \sum_{i=1}^m A_i(r + (p-1)n')\alpha_i^{r} (\alpha_i^{(p-1)})^{n'}
\end{equation*}
for each fixed \(r\) this function of \(n'\) is analytic.
\pause Infinitely many zeroes for integer \(n\) means \(\exists r\) with infinitely many zeroes of the form \(r + (p-1)n'\). So the function%
\begin{equation*}
\sum_{i=1}^m A_i(r + (p-1)n')\alpha_i^{r} (\alpha_i^{(p-1)})^{n'}
\end{equation*}
is identically zero, and all these \(a_n = 0\) when \(n \equiv r \pmod{p-1}\).%
\par
\end{frame}

\begin{frame}{Finale}
\begin{theorem}[{Skolem \(\leadsto\) Mahler \(\leadsto\) Lech}]
All except finitely many indicies of the zeroes of a linear recurrence lie in a a finite union of arithmetric progressions, i.e.\ they are all of the form \(nM + b\) for some \(b \in B \subset \{0, \ldots, M-1\}\), \(n \in \NN\).%
\end{theorem}
\pause
\includegraphics[height = 0.6\textheight]{skolem.jpeg}
\end{frame}

\end{document}
