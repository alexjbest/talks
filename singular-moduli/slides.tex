\documentclass{beamer}
%\documentclass[handout]{beamer}
%\documentclass[handout,notes=show]{beamer}

\usetheme{Frankfurt}
\usecolortheme{dolphin}

\usepackage{amsmath, amssymb, amsfonts, tikz}
\usepackage[utf8]{inputenc}
\usepackage[T1]{fontenc}
\usepackage[english]{babel}

\DeclareTextFontCommand{\emph}{\bfseries}

\author{Alex J. Best}
\institute{WIMP 2014}
\date{29/11/2014}
\title{Singular Moduli}

\begin{document}

\section{Introduction}

\frame{\titlepage
\note{
Although I am no longer at Warwick I was here for a few years and I'd like to thank the organisers for giving me the excuse to come back and see many old friends, but also for chance to tell you about this very interesting topic.
}
}

\begin{frame}
\frametitle{In this talk:}
\note{
Will talk about some special values of a special function, their properties and the surprising consequences.
}
\tableofcontents
\end{frame}

\section{Background}

\begin{frame}{}
\note{
After Alex's (excellent?) talk will go easy on the background, but if anything remains unclear please do stop me and ask.
}
\begin{itemize}
\item .
\pause \item .
\end{itemize}
\end{frame}

\section{Conclusion}
\begin{frame}{Closing remarks}
\begin{itemize}
\item Singular moduli are not particularly complex objects in and of themselves.
\pause \item But their relation between different areas of mathematics ensures that they are still a research topic to this day.
\end{itemize}
\end{frame}

\begin{frame}{Sources}
I used some of the following when preparing this talk, and so they are possibly good places to look to learn more about the topic:
\begin{enumerate}
\item ``Primes of the form $x^2 + ny^2$'' -- David A. Cox
\end{enumerate}
\end{frame}

\end{document}


