\documentclass{beamer}
%\documentclass[notheorems]{beamer}
% \documentclass[handout]{beamer}
%\documentclass[handout,notes=show]{beamer}

\usetheme{metropolis}

\usepackage{amsmath, amssymb, amsfonts, tikz}
\usepackage[utf8]{inputenc}
\usepackage[T1]{fontenc}
\usepackage[english]{babel}

%\usepackage[default,osfigures,scale=0.95]{opensans}


%\usepackage{fontspec}
%\usefonttheme{professionalfonts} % using non standard fonts for beamer
%\setsansfont[
%Path           = /Users/alex/Library/Fonts/,
%Extension      = .ttf,
%Ligatures      = TeX,
%UprightFont    = OpenSans-Light,
%BoldFont       = OpenSans-Regular,
%ItalicFont     = OpenSans-LightItalic,
%BoldItalicFont = OpenSans-Italic
%]{OpenSans}
%\setmonofont[
%Path           = /Users/alex/Library/Fonts/,
%Extension      = .ttf,
%Ligatures      = TeX,
%UprightFont    = FiraCode-Regular,
%ItalicFont    = FiraCode-Regular,
%ItalicFont    = FiraCode-Regular,
%]{DejaVu Sans Mono}
\metroset{block=fill}

% No navigation bars 
\beamertemplatenavigationsymbolsempty

\graphicspath{{img/}}

\definecolor{Purple}{HTML}{911146}
\definecolor{Orange}{HTML}{CF4A30}
\definecolor{Tan}{RGB}{225,221,191}
\definecolor{Green}{RGB}{76,131,122}
\definecolor{DB}{RGB}{4,37,58}

% Theme colors are derived from these two elements
\setbeamercolor{alerted text}{fg=Green}
\setbeamercolor{frametitle}{bg=Tan,fg=DB}
\usepackage{tabularx}
\usepackage{mathtools}
\usepackage{xmpmulti}
\usepackage{color}
\definecolor{keywordcolor}{rgb}{0.7, 0.1, 0.1}   % red
\definecolor{commentcolor}{rgb}{0.4, 0.4, 0.4}   % grey
\definecolor{symbolcolor}{rgb}{0.0, 0.1, 0.6}    % blue
\definecolor{sortcolor}{rgb}{0.1, 0.5, 0.1}      % green
\metroset{titleformat=smallcaps}


\newtheorem{proposition}[theorem]{Proposition}
\newtheorem{remarks}[theorem]{Remarks}
\newtheorem{remark}[theorem]{Remark}
\newtheorem{conjecture}[theorem]{Conjecture}


\theoremstyle{plain}
\newcommand{\terminology}[1]{\textbf{#1}}

\newcommand{\NN}{\mathbf{N}}
\newcommand{\ZZ}{\mathbf{Z}}
\newcommand{\QQ}{\mathbf Q}
\newcommand{\CC}{\mathbf C}
\newcommand{\RR}{\mathbf R}
\newcommand{\FF}{\mathbf F}
\newcommand{\lt}{<}
\newcommand{\gt}{>}
\newcommand{\amp}{&}
\newcommand{\diff}{\mathop{}\!\mathrm{d}}
\newcommand{\ints}{\mathcal{O}}
\newcommand{\ideal}[1]{\mathfrak{#1}}
\usepackage{mathrsfs}\usepackage{cancel}
\newcommand{\Gal}[2]{\operatorname{Gal}(#1/#2)}
\newcommand{\absgal}[1]{\operatorname{Gal}(\overline{#1}/#1)}
\DeclareMathOperator{\USp}{USp}
\DeclareMathOperator{\Spec}{Spec}

\newcommand{\sheaf}[1]{\operatorname{\mathcal{#1}}}
\newcommand{\inv}{^{-1}}
\DeclareMathOperator{\norm}{Nm}
\DeclareMathOperator{\ord}{ord}
\DeclareMathOperator{\divisor}{div}
\DeclareMathOperator{\PP}{\mathbf{P}}
\DeclareMathOperator{\Hom}{Hom}
\DeclareMathOperator{\Mat}{Mat}
\DeclareMathOperator{\End}{End}

\newcommand{\lb}{[}
\newcommand{\rb}{]}


\usepackage{listings}
\def\lstlanguagefiles{tex/lstlean.tex}

\lstset{language=lean,basicstyle=\ttfamily\scriptsize}

%\usepackage{mathfont}


\author{Alex J. Best}
\date{9/4/2024}
\title{Formalization and Arithmetic Geometry}
\subtitle{past, present, and future}

\begin{document}

\begin{frame}
  \titlepage

  \note[item]{Thank the audience for being awake.}
\end{frame}

%\begin{frame}
%\frametitle{Table of Contents}
%\tableofcontents[currentsection]
%\end{frame}

\begin{frame}{Formalization}
    % i'm working in an area 
    Expressing mathematics (objects, arguments) in a format that a computer can handle and interact with rigorously.

    \only<2-4>{Some examples of the state of the art: }

    \only<2>{\includegraphics[width=1.05\linewidth]{pfr.png}}
    \only<3>{\includegraphics[width=1.05\linewidth]{pfr2.png}}
    \only<4>{\includegraphics[width=1.05\linewidth]{patrick.png}}
\end{frame}

% TODO use speaker notes, print

\begin{frame}{Potential use cases of formalization}
    % i'm working in an area 
    \begin{itemize}
    \item Quickly searching for previously formalized results that may be useful in a given situation
    \item Automation of routine arguments, letting the software worry about the details

    \item Producing documents for which we can easily look up precise definitions, or tell halfway through a paper what the current objects being talked about are\note{, what assumptions on them we have at that point}

    \item Producing interactive documents where the user can choose the level of detail they want to see, or the shortest path to understand a given result.
    \end{itemize}

\end{frame}
\begin{frame}{Potential use cases of formalization, cont}

    \begin{itemize}
    \item Error free or higher confidence in the details of published mathematics

    \item Can lead to bug free mathematical software, verified plotting
    \item Allow easier modification of previously formalized material

    \item
        Machine learning and AI; some types of machine learning based tools can already be helpful when formalizing.
        Can they they eventually produce a page of mathematics given a brief prompt? Can they big ideas autonomously? Even being able to check that routine arguments similar to those in the literature hold would be very useful.
    \end{itemize}

    \emph{Would like to have as many of these as possible without sacrificing benefits of existing presentation methods; readability and easy comprehension for trained people (any mathematician) and ease of writing.
    Not only should this technology make it easier for computers to do maths, but ideally also humans.}
    \note{no more appeal to this argument is similar to the one in x paper, but likely when writing that you already check the details once yourself}
\end{frame}

\begin{frame}{Demo time!}
    Lets see how proving a result using a proof assistant (the most common way of formalizing advanced mathematics) looks.

    We will prove Euclid's theorem using Mathlib, a large library of formalized mathematics in Lean, that has attracted a community of mathematicians as developers and users.
\end{frame}

\begin{frame}{Some recent high profile examples of mathematical formalization projects}
    % i'm working in an area 
    \begin{itemize}
        \item Tao led a project to formalize a proof of  the Polynomial Freiman-Rusza conjecture (Gowers-Green-Manners-Tao).
            Finished 3 weeks! 25 contributors, analogous to a long reading group / summer school.
        \item Scholze challenged the formalization community to formalize 
            key result in liquid condensed mathematics (Clausen-Scholze). This took a group of up to 25 people a year and a half. \note{Dagur working on solid}
        \item Dillies Mehta and Bloom formalizing results in additive combinatorics in ``real time'' %Goeuzel - Birkhoff ergodic theorem, Gromov-Hausdorff measures
        \item Massot, van Doorn and Nash - Gromov's $h$-principle and sphere eversion
        \item Buzzard, Commelin and Massot, formalizing the definition of a perfectoid space.
        \item Formal proof of the Kepler conjecture
        \item $\cdots$
    \end{itemize}
    \note{projcets getting more collaborative}
\end{frame}

% name drop some other mathematicians

\begin{frame}{Future goals}
    \begin{itemize}
        \item
    Strong evidence that with effort most modern mathematical results can be formalized.
    Challenge is to make this not just a one-off, but a sustainable process that doesn't require as many person-hours as the above projects took.
    Can come from finding efficient mathematical arguments but also improving the formalization language and surrounding tools. To be closer to the level of abstraction we are used to. \pause
    % i'm working in an area 

        \item
    Standard undergraduate curriculum is close to all already formalized.\note{handy for teaching}
            \note{ Doesn't mean it is  ``done'' per se, people still wrote textbooks after Bourbaki!
            Much of this material is still formalized in a way that doesn't match how we would ideally interact with these systems.} \pause

        \item
    Goals for future formalization projects are to consider, higher level arguments, areas less obviously formal, those with more appeals to intuition or unverified computation, or sheer volume of very technical material or techniques.
    Can most areas of mathematics be conveniently expressed in a proof assistant?
    %Not all about correctness, but about being able to do something useful with the outputs.
    \end{itemize}
\end{frame}

\begin{frame}{Arithmetic Geometry}
    Formalization and arithmetic geometry sometimes feels less well developed than many other fields, why?:
    \begin{itemize}
        \item Requires a lot of theory to get to some basic tools, scheme theory, cohomology theories \pause
        \item social reasons, researchers in arithmetic geometry interested in formalization ended up working on projects like Liquid Tensor Experiment \note{played a big role in developing eg cohomology but not directly} \pause
        \item ... maybe its not true at all? \note{maybe im biased}
    \end{itemize}
\end{frame}

\begin{frame}{Some general results of interest that have been formalized}
    \begin{itemize}
        \item Local and global fields
        \item Algebraic number theory tools, finiteness of class group, Dirichlet's unit theorem, Kummer-Dedekind
        \item (Absolute) Galois groups (and some cohomology thereof)
        \item Ideles and Adeles
        \item Witt Vectors (and $p$-adic fields)
        \item $L$-series, and modular forms
        \item $p$-adic $L$-functions
        \item Ostrowski's theorem (again at LFTCM last week!)
        \item Divided power structures (towards $B_{dR}$)
        \item Schemes
        \item Elliptic Curves, group law in all characteristics
    \end{itemize}
    Roblot, de Frutos Fernandez, Baanen, Dahmen, Narayanan, Nuccio, Chambert-Loir, Loeffler, Stoll, Commelin, Lewis, Livingston, Birkbeck, etc

\end{frame}


\begin{frame}{In the (near) future}
    Alex Kontorovich and Terrence Tao have started a project to formalize the PNT+ project, to formalize the Prime Number theorem in Lean, and other related results such as Chebotarev and Dirichlet's theorem on primes in arithmetic progressions.

    We will hear more on this from either Michael Stoll or Alex Kontorovich in this seminar!

    Chebotarev is a vital technical tool for Arithmetic Geometry, underpinning many important results and techniques so this is likely to be very useful.\pause

    Kevin Buzzard is starting a longer term project to formalize much of the mathematics around Fermat's Last Theorem

\end{frame}

\begin{frame}{Algebraic number theory: FLT-regular}
    With Birbeck, Brasca, Rodriguez, Yang we formalized the proof due to Kummer of Fermat's last theorem for regular primes.
    \note{late 2023}


    \begin{itemize}
        \item It pays to take the time to find the right proof, a proof reducing the main technical tool, Kummer's Lemma, to Hilbert's theorem 92 avoiding CFT was in the exercises of Swinnerton-Dyer's textbook!

            \only<2>{\textbf{Kummer:} Let $p$ be a regular prime and let $u \in \mathbb{Z}\left[\zeta_p\right]^{\times}$. If $u \equiv a \bmod p$ for some $a \in \mathbb{Z}$, then there exists $v \in \mathbb{Z}\left[\zeta_p\right]^{\times}$such that $u=v^p$.

            \textbf{Hilbert:} Let $K / F$ be a Galois extension of $F=\mathbb{Q}\left(\zeta_p\right)$ with Galois group $\operatorname{Gal}(K / F)$ cyclic with generator $\sigma$. Then there exists a unit $\eta \in \mathcal{O}_K$ such that $N_{K / F}(\eta)=1$ but does not have the form $\epsilon / \sigma(\epsilon)$ for any unit $\epsilon \in \mathcal{O}_K$.

            See: \url{https://leanprover-community.github.io/flt-regular/blueprint/}}

        \item<3> Required developing a theory of cyclotomic fields and rings, its much easier to work with abstract generality than with the model you actually need.
    \end{itemize}
\end{frame}

\begin{frame}{Basic Diophantine equations}
    With Baanen, Coppola and Dahmen (2023) we wanted to try some classic examples of arithmetic geometry: determining explicitly the integral points on some elliptic curves. E.g.
    $$y^2 = x^3 - 5, y^2 = x^3 - 17$$
    this is via Mordell-style descent\pause

    Lessons learned:
    \begin{itemize}
        \item
    theory can be easier than calculation, when calculating we tend to make a lot more steps implicitly
    e.g. when saying the product of two explicit ideals in some number field is nontrival in class group\pause
    
        \item
    Having the system automatically do calculations in explicit rings (given by a times table, or in positive characteristic), is a massive help, as is being able to add this sort of of functionality as a user is essential. 
    \end{itemize}
\end{frame}

\begin{frame}{Certification}
    Lessons learned (cont):
    \begin{itemize}
        \item
            More generally importing results from computer algebra systems will be useful when formalizing explicit type results.\pause
        \item
    How do you ``certify'' a class or unit group computation? What is the shortest/simplest proof that a given 
    class group is what is claimed assuming that checking takes a lot more work than finding. With the analytic class number formula this becomes easier, but then leads to how to do numerics for $L$-functions with verified bounds.\pause

        \item
    Doing things in a low-tech way can lead you to look at some very pretty mathematics.
    Previously I would never have thought about how to bound class groups of quadratic fields with anything other than Minkowski, but it turns out there is a fascinating connection to the Farey sequence there.
    \end{itemize}
\end{frame}

\begin{frame}{Tate's algorithm}
    Sacha Huriot-Tattegrain (+B.+Dahmen) has implemented Tate's algorithm in Lean(4).

    \begin{itemize}
        \item Complete algorithm to compute local invariants of an elliptic curve, including the $c_p(E), \ord_p(\Delta_E), \ord_p(N_E)$

        \item Works in characteristic 2 and 3.

        \item Based on Cohen's description of the algorithm, but at times consulting other sources and even the GP source code was necessary to get it right.

        \item It runs fast!

        \item Partly generalized to base rings beyond $\ZZ$.
    \end{itemize}

    Without an independent definition of the Kodaira types and conductor exponent we cannot actually check the algorithm does what it says.
    But we could prove certain properties of the algorithm in future, such as invariance under change of model.
\end{frame}

\begin{frame}{Thanks for listening}
    Formalization of mathematics (including number theory) is still slow and painful at times.

    But we have several thousand years of mathematics, and learning how to think about, and explain mathematics, to catch up on.

    Thinking about these issues and finding nice arguments and general techniques can be a lot of fun, and the tool may occasionally surprise you.

    If you are interested in getting more actively involved check out 
    \url{https://leanprover-community.github.io/}
    and
    \url{https://leanprover.zulipchat.com/}
    also see if there is a ``Lean for the curious mathematician'' or even a ``Lean for the curious arithmetic geometer'' near you.
\end{frame}


%\begin{frame}{My research: formalizing mathematical theory}
%    With various collaborators I have been working on formalizing mathematics, particularly algebraic number theory, and the basics some initial steps in arithmetic geometry.
%    Some of the large projects I've played a significant role in are:
%    \begin{itemize}
%        \item Minkowski theory.
%        \item Explicit class group computations.
%        \item Determining integral points on Mordell curves via descent.
%        \item Proving Fermat's last theorem for regular primes.
%        \item Formalizing Tate's algorithm for local invariants of elliptic curves.
%    \end{itemize}

%    Each of these projects involved the creation of new tools, or improvements to existing ones, to make formalizing further material in related areas easier.
%\end{frame}

\newcommand{\comment}[2]{#2}

\comment{
\begin{frame}{My research: unlocking benefits of formalized mathematics}
    Rather than formalizing more and more material, focussed on improving usability of proof assistants, and exploring potential applications.

    For example: if we formalize proof under some assumptions we know are sufficient, can the proof \emph{assistant} software tell us the most general assumptions that our proof makes sense under?
\pause
    Consider:
    \begin{lemma}
        Let $f\colon K\to L$ be a ring homomorphism between two fields, and $p$ be a natural number, then $K$ is characteristic $p$ if and only if $L$ is (including $p= 0$).
    \end{lemma}
\pause
    I demonstrated that just by inspecting the original formalized proof there is a procedure to extract more general assumptions from the given formalized proof.

    \note{Applying this for the above a piece of code in the proof assistant will happily inform us that $K$ can be any division ring and $L$ can in fact be any nontrivial semiring.}
\end{frame}

\begin{frame}{My research: present/future}
    \begin{itemize}
        \item Improving upon links between formalized mathematics and computer algebra systems. To allow computational arguments to be more easily included
              in other formalization projects. And also to allow formalized mathematical definitions and results to be used in computer algebra systems.
        \item Implementing ways to internalize already formalized material inside a topos in the same formal system. This involves taking meta-mathematical ideas coming from logic and producing practical tools that make formalization more convenient for mathematicians.
    \end{itemize}
\end{frame}
\begin{frame}{Thank you for your attention}
    Questions?
\end{frame}

\begin{frame}
\end{frame}


also cover importance of number theoretic algorithms

# Past
- Kepler - one big proof
- Elementary nt, via custom rings, pell etc
- Witt vectors
- Perfectoid spaces
- insolubility of the quintic in coq / lean,
- schemes in Lean (+ Isabelle too),
- some of the bits I did with Minkowski (also in HOL + isabelle),
- mason-stothers,
- basics of arith fns
- Lindemann-Weierstrass in coq
- ostrowski.
- prime number theorem
  - should be possible to "port" such a theorem to lean, but whats the point
- 1-dim diuedone

# Present 
## Some assorted smaller projects in mathlib
- David Loeffler adding Gamma function
- Kevin wilson density of squarefree numbers
- Michael Stoll, playing with hilbert symbols
- Antoine Chambert-Loir: Finite groups, simplicity of A_n's,
    A primitive subgroup of a permutation group \mathfrak S_nS
    n that contains a transposition is equal to \mathfrak S_nS
    n ; a primitive subgroup that contains a 3-cycle contains the alternating group.
  working from a group theory textbook Weilandt
  (maybe interesting to note that he was member of bourbaki until his 50th birthday in 2021,
  shortly thereafter started working more on some lean projects
- Alex kontorovich
- riccardo
- nuccio

- modular forms
- elliptic curves
- Sacha lean4, based on cohen, compare with, rings vs Z
- flt-regular, - mention computation of bernoulli in isabelle using...
- the unit fractions project,
- descent and the Sage bridge,
- adeles and statements of cft,
- p-adic L-functions
- Johan's talk

# Future
Issues / problems of interest
- Main issue is making the tools easier to use:
  - less verbose
  - tooling to catch common mistakes and problems, maybe even mathematically useful.
- writing at a level of detail closer to pen and paper maths, when the results are actually easy or standard in some way.
- how to speed up such 



I'd also like to mention some of the tooling / lintery things I wrote
and maybe find some nice instances where that's helped people.

I'd like to actually talk to a couple of people this week and find out
what they'd most like to hear, it's hard at this point to know what
people unfamiliar with the area are most interested in.

\begin{frame}{In the 2000s}
    Large projects such as the Kepler conjecture and the Odd order theorem.

    These were big collaborations with one main goal, and did involve some number theory adjacent topics.
    
    Now there is more of a trend to build on existing libraries to make more progress on deeper topics.
\end{frame}
}


%\begin{frame}{Lindemann-Weierstrass}
%    In Coq and Isabelle
%\end{frame}

\comment{
\begin{frame}[fragile]{pp}
\begin{lstlisting}
lemma fact_rec (n : ℕ) :
factorial (n + 1) = factorial n * (n+1) :=
begin
-- write out the definition of factorial
unfold factorial,
-- remember {1,...,n+1} = {1,...,n} ∪ {n+1}
rewrite list.range'_concat 1 n,
-- the product of two sequences joined together is just the product of the products of each sequence
rewrite list.prod_append,
-- I'm bored already are we done here?
simp,
-- YES!
end
\end{lstlisting}\pause

\end{frame}}


\begin{frame}{Perfectoids}
    Peter Scholze won a Fields medal in 2018 for ``transforming arithmetic algebraic geometry over $p$-adic fields through his introduction of perfectoid spaces, with application to Galois representations, and for the development of new cohomology theories.''
    \pause
    The definition is highly nontrivial, an unusual geometric object created from an extremely non-Noetherian ring.
    \pause

    \emph{In 2020ish} Kevin Buzzard, Johan Commelin, Patrick Massot  (building on others)  completed a long term project to define a perfectoid space formally in Lean.
\end{frame}

\begin{frame}{Perfectoids}
    \includegraphics[width=\textwidth]{perfectoid.png}
    Lean has accepted the chain of definitions that lead to this are all valid, topological spaces, sheaves, valuations, adic spaces, perfectoid rings,...
    \pause

    It is difficult to estimate the amount of human effort expended to achieve this. \pause
    Their work relied on that of many others who are building \texttt{mathlib}, a general purpose library of mathematics from the ground up. \pause

    However, it also takes a long time for a human with no mathematical background to learn such a definition.
\end{frame}

\begin{frame}{One side effect: "new" algebraic structures}
    One ingredient of the theory surrounding perfectoid spaces (adic, spectral, Huber rings, etc.) is the notion of a valuation
    $$K \to \Gamma \cup \{0\}$$
    sending 0 to 0.

    In the course of the project the authors noticed they were having to repeat a lot of work on basic lemmas that were true both for fields and the value group above, inspired the creation of a new definition, a \emph{group with zero} (and monoid with zero, etc.).

    \begin{quote}
    ``Every sufficiently good analogy is yearning to become a functor.'' -- John Baez
    \end{quote}

    \begin{quote}
    Every sufficiently similar proof is yearning to become a new algebraic structure.
    \end{quote}
\end{frame}

\begin{frame}{Niche algebraic structures}
    There is even a lot of duplication between lemmas about groups, and those about groups with a zero.

    Earlier this year Yaël Dillies introduced a new algebraic structure, a division monoid, to be the correct setting for theorems, this is a monoid with an involutive inverse operation that doesn't always have $a \cdot a\inv  = 1$, but does have $a \cdot b = 1$ implies $a \inv = b$.

    \textbf{Upshot:} In order to formalize effectively and reduce duplication of effort generalizing proofs to unfamiliar algebraic structures is helpful.

\end{frame}


\begin{frame}{Backing up}
    Despite there being impressive progress on very advanced number theory, at least in the mathlib library there was not even the definition of a number field in Lean at the time

    Baanen, Dahmen, Ashvni Narayanan, Filippo Nuccio added Dedekind domains, and proved finiteness of the class group last year.

    Interestingly this formalization is uniform in the number field and function field cases, and avoids Minkowski's theorem in favour of simpler pigeonhole-type principles.

    But the basics of algebraic number theory are not really complete (Kummer-Dedekind, Kummer theory, Kronecker-Weber) in any formal system that I know.
\end{frame}


\begin{frame}{Some progress}
    María Inés de Frutos Fernández has formalized the ring of Adèles (and Idèles) and given the \emph{statement} of the main theorem of global CFT in Lean:
    \begin{theorem}
        Let $K$ be a number field. Denote by $C_{K}^{1}$ the quotient of $C_{K}$ by the connected component of the identity. There is an isomorphism of topological groups $C_{K}^{1} \simeq G_{K}^{a b}$.
    \end{theorem}
    \includegraphics[width=\textwidth]{maria.png}
\end{frame}

\begin{frame}{Descent}
    With Anne Baanen, Nirvana Coppola, Sander Dahmen, we have been formalizing some Mordell-style descent to find integral points on elliptic curves: for example the non-existence of integral points on
    $$y^2 = x^3 - 5$$\pause
    Basically works, except, we still need to compute the class group of $\QQ(\sqrt{-5})$!

    This sort of proof necessarily involves some amount of hands on calculation, this is often harder to formalize than clean theory. % Part of this is due to implicit normal forms in the way we think about elements of number fields, writing everything formally in terms of a basis is annoying!

    In order to work conveniently with such calculations we have added tactics to handle calculations in rings with a finite ``multiplication table'' automatically, and write formal proofs that aren't significantly longer than paper ones.

    The other strategy is to leverage existing computer algebra systems where possible but still checking the output.
\end{frame}

\begin{frame}{Certifying number theoretic computations}
    Eventually would be helpful to have code that computes class groups implemented in a formal system.

    Right now this is a lot of work repeating the excellent pre-existing algorithms in a new language.  \pause

    \textbf{Question:} Is it possible to compute the class group with a computer algebra system (e.g. Sage), and write down a certificate of the result that is easily checkable (fast to check, not too long, and mathematically simple!)

    Ideally the certificate would be a text file, other users shouldn't need to install the CAS to repeat the calculation, but it should be provable in the system.

    But the certification itself should not rely on GRH etc.
\end{frame}

\begin{frame}{The Hasse Norm theorem}
    Suppose we want to check that an explicitly given ideal in a number field is non-principal, can we give a certificate for this.

    One idea: If an ideal is principal, it's norm must be equal to the norm of an element (and this holds everywhere locally too).

    \begin{theorem}[Hasse Norm theorem]
        If $K/\mathbf Q$ is a cyclic Galois extension and $x \in \mathbf Q$ is everywhere locally a norm, then $x$ is globally a norm.
    \end{theorem}
\pause
    There are counterexamples to this in the biquadratic case due to Hasse (and Serre-Tate) (and for any non-cyclic case Frei, Loughran, Newton).

    Number fields for which this property holds are said to satisfy the \emph{Hasse norm principle}.
\end{frame}

\begin{frame}{The Hasse Norm theorem}
    \begin{theorem}[Frei, Loughran, Newton]
        Let $k$ be a number field and $G$ a finite abelian group. Then 100\% of $G$-extensions of $k$, ordered by conductor, satisfy the Hasse norm principle.
    \end{theorem}\pause
    But if we order by discriminant:

    \begin{theorem}[Frei, Loughran, Newton] Let $G$ be a non-trivial finite abelian group and let $Q$ be the smallest prime dividing $|G|$. Assume that $G$ is not isomorphic to a group of the form $\mathbb{Z} / n \mathbb{Z} \oplus(\mathbb{Z} / Q \mathbb{Z})^{r}$ for any $n$ divisible by $Q$ and $r \geq 0$. Then a positive proportion of $G$-extensions of $k$ fail the Hasse norm principle, ordered by discriminant.
    \end{theorem}
    So locally verifying non-principality might be viable for abelian number fields.
\end{frame}

\begin{frame}{Other ideas}
    There are many useful algorithms with "obvious" certificates:
    \begin{itemize}
        \item Ideal membership
        \item Matrix normal forms (SNF, HNF, LU, RREF)
        \item Factoring
        \item Checking solubility modulo primes
    \end{itemize}
\pause
    Have a tool that talks to Sage to certify some of these in Lean already, working on others.

    I'd be happy to learn of other instances of this pattern!

    This might be independently a nice check for CASes, when further advanced.
\end{frame}

%\begin{frame}{Implementing number theoretic algorithms}
%\end{frame}

\begin{frame}{Implementing number theoretic algorithms}
    Alternatively we can implement algorithms within a proof assistant, as efficient functions that give the same output as what we want to compute
    \begin{itemize}
        \item Gives us a guaranteed correct implementation.
        \item We can experiment with modifying / improving the algorithm, and prove correctness or equality with the original one.
        \item We can prove properties, or "run" the algorithm in families, in ways normal code can't.
    \end{itemize}

    After writing the algorithm down, it is only accepted as a genuine mathematical function when it is shown to halt.
    With some functions this is obvious, but for algorithms that use recursion or unbounded loops, less so!
\end{frame}



\begin{frame}{Unit fractions}
    In December 2021 Thomas Bloom posted a paper: On a Density Conjecture about Unit Fractions to arXiv (2112.03726)
    \begin{quote}
        \textbf{Abstract:} We prove that any set $A \subset \mathbb{N}$ of positive upper density contains a finite $S \subset A$ such that $\sum_{n \in S} \frac{1}{n}=1$, answering a question of Erdős and Graham.
    \end{quote}
    18 pages, quickly recognized as correct and widely applauded in popular press (Quanta, etc), generalizes an older result of Croot.

    Thomas Bloom and Bhavik Mehta are working hard to formalize the paper.
\end{frame}

\begin{frame}
    \includegraphics[width=1.1\linewidth]{unit.png}

    Many nice outputs from this project for analytic number theory and density results too.
\end{frame}

\begin{frame}{Collaboration Galore}
    One nice aspect of formalization is community, we are building on each others work, but the gaps have to line up precisely.

    This both eases collaboration (I can not worry about the details of your proof if it compiles and I understand the statement), but it also makes it harder, I have to contend and work with the community agreed upon definition of an object, rather than make my own variant.

    Nevertheless working on such a library has the feeling of collaborating on a large textbook / reference work.
\end{frame}

\begin{frame}{Collaboration Galore}
    \begin{itemize}
        \item Chris Birkbeck: Defining modular forms + Eisenstein series (like Manuel!)
        \item David Loeffler: Defining the Gamma function, analytic continuation
        \item Antoine Chambert-Loir: Finite groups, simplicity of $A_n$'s
        \item Amelia Livingston: Group cohomology
        \item Brandon H. Gomes and Alex Kontorovich: statement of the Riemann Hypothesis
        \item Michael Stoll: re-doing Legendre symbols, proved Hilbert reciprocity for quadratic Hilbert symbols over $\QQ$
        \item Sophie Bernard \& Cyril Cohen \& Assia Mahboubi \& Pierre-Yves Strub, and Thomas Browning: Insolvability of General Higher Degree Equations
        \item Kevin Wilson: calculation of the density of squarefree numbers as $\zeta (2)\inv = 6 / \pi ^2 $.
    \end{itemize}
\end{frame}

\begin{frame}{Closing thoughts}
\end{frame}


\comment{








\begin{frame}{Some examples: Grunwald(-Wang) and K-theory}

    \begin{quotation}
        Some days later I was with Artin in his office when Wang appeared. He said he had a counterexample to a lemma which had been used in the proof. An hour or two later, he produced a counterexample to the theorem itself... Of course he [Artin] was astonished, as were all of us students, that a famous theorem with two published proofs, one of which we had all heard in the seminar without our noticing anything, could be wrong.

        --- Tate
    \end{quotation}\pause

    \note{The problem: 2, the cursed prime. This is often an edge case.}

    \begin{quotation}
        The groundbreaking 1986 paper “Algebraic Cycles and Higher K-theory” by Spencer Bloch was soon after publication found by Andrei Suslin to contain a mistake in the proof of Lemma 1.1. The proof could not be fixed.

        --- Voevodsky
    \end{quotation}

\end{frame}

\begin{frame}{The solution: Take 2}
    What if computers could do the boring work for us?\pause

    Computers are:
    \begin{itemize}
        \item Capable of checking basic logical statements,\pause
        \item Fast,\pause
        \item Never complain.
    \end{itemize}
\end{frame}

\begin{frame}{The new problem}
    How do you describe the steps of a proof to a computer with as little pain as possible? \pause
    Often mathematicians leave unsaid many steps which are intuitive or easily supplied. \note{(this is one place mistakes enter)}\pause

    \includegraphics[height=2em]{pf-exercise.png}\pause

    \includegraphics[height=1.5em]{beweis-klar.png}\pause

    \includegraphics[height=1.5em]{pf-similar.png}\pause

    \includegraphics[height=2em]{pf-obvious.png}\pause

    \includegraphics[height=1.5em]{pf-remember.png}\pause

    \includegraphics[height=1.1em]{pf-easy-induct.png}\pause

    The computer will probably not understand these, but in order to stay sane we must strike a balance between detail and verbosity.\note{ Teach the computer to work off as little as possible.}
\end{frame}

\begin{frame}{Enter Lean}
    The idea of trying this sort of thing has been around for a while. \pause

    But only within the last few years has it begun to seem more feasible for an average mathematician to do this. Tools have gotten better, slowly this idea has gained traction.\pause

    Recently an interactive proof assistant called Lean has been under heavy development. And I've been playing with it.

    Let me show you some lean code:
\end{frame}

\begin{frame}[fragile]
\begin{lstlisting}
lemma fact_rec (n : ℕ) :
factorial (n + 1) = factorial n * (n+1) :=
begin
-- write out the definition of factorial
unfold factorial,
-- remember {1,...,n+1} = {1,...,n} ∪ {n+1}
rewrite list.range'_concat 1 n,
-- the product of two sequences joined together is just the product of the products of each sequence
rewrite list.prod_append,
-- I'm bored already are we done here?
simp,
-- YES!
end
\end{lstlisting}\pause
We can replace all of the above with: \lstinline{by unfold factorial; simp [list.range'_concat, list.prod_append]}
\note{ Lean will figure out when and how to apply the lemmas.}

\end{frame}

\begin{frame}{(not so?) Live demo!}
    \multiinclude[<+->][format=png, graphics={width=\textwidth}]{fact}
\end{frame}

\begin{frame}{But can it do research?}
    Research level mathematics requires a vast body of knowledge to even think about.
    \pause

    As humans we forget this and also gloss over things we (think we) know well.
    \pause

    For instance one can forget what a Dedekind cut is, or Peano arithmetic and still think about real and natural numbers without an issue.
    Our intuition allows us to abstract these concepts so far away that we don't have to work from the ground up when approaching a problem.

\end{frame}}

\end{document}
