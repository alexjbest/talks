\documentclass[12pt]{amsart}

\usepackage{latexsym}
\usepackage{amsmath}
\usepackage{amssymb}
\usepackage{epsfig}
\usepackage{amsfonts}
\usepackage{amscd}
\usepackage{mathrsfs}

\newtheorem{theorem}{Theorem}
\newtheorem{lemma}[theorem]{Lemma}
\newtheorem{claim}[theorem]{Claim}
\newtheorem{cor}[theorem]{Corollary}
\newtheorem{prop}[theorem]{Proposition}
\newtheorem{conj}[theorem]{Conjecture}
\newtheorem{defn}[theorem]{Definition}

%%%%%%%%%%%%%%%%%%%%%%%%%%%%%%%%%%%%%%%%%%%%%%%%%%%%%%%%%%%%%%%%%%%%%%%%
%%%%%%%%%%%%%%%%%%%%%%%%%%%%%%%%%%%%%%%%%%%%%%%%%%%%%%%%%%%%%%%%%%%%%%%%
%%%%%%%%%%%%%%%%   PLEASE DO NOT EDIT ABOVE THIS LINE   %%%%%%%%%%%%%%%%
%%%%%%%%%%%%%%%%%%%%%%%%%%%%%%%%%%%%%%%%%%%%%%%%%%%%%%%%%%%%%%%%%%%%%%%%
%%%%%%%%%%%%%%%%%%%%%%%%%%%%%%%%%%%%%%%%%%%%%%%%%%%%%%%%%%%%%%%%%%%%%%%%

%%%%%%%%%%%%%%%%%%%%%%%%%%%%%%%%%%%%%%%%%%%%%%%%%%%%%%%%%%%%%%%%%%%%%%%%
\begin{document}%%%%%%%%%%%%%%%%%%%%%%%%%%%%%%%%%%%%%%%%%%%%%%%%%%%%%%%%
%%%%%%%%%%%%%%%%%%%%%%%%%%%%%%%%%%%%%%%%%%%%%%%%%%%%%%%%%%%%%%%%%%%%%%%%


\title{Explicit computation with Coleman integrals and applications}

\author{Alex J. Best (*)}
\address{Boston University}
\email{alexbest@bu.edu}

\subjclass[2010]{Primary 11G20; Secondary 11Y16, 14F30}

\keywords{Coleman integration, superelliptic curves, Kedlaya's algorithm}

\date{May 2019}

\begin{abstract}
\noindent


Coleman integration is a \(p\)-adic integration theory that provides a
way integrate 1-forms on curves over \(p\)-adic fields \cite{bib-coleman-torsion}. The applications
of this theory to arithmetic questions are varied, but include
definitions of \(p\)-adic polylogarithms, explicit determination of
rational points and torsion points on curves, and the computation of
regulator maps in \(K\)-theory.

We will discuss the problem of computing these integrals in practice for
special classes of interesting curves, such as hyperelliptic or superelliptic curves. Via Monsky-Washnitzer cohomology
this problem is related to that of computing zeta functions of curves
over finite fields \cite{bib-balakrishnan-bradshaw-kedlaya}. One can try to extend methods for computing the zeta
function to also compute Coleman integrals. Several algorithms
can be developed for Coleman integration in this way \cite{bib-best-coleman-harvey,bib-best-super} with advantages and disadvantages of each. We will also mention work
in progress to use such computations as part of quadratic Chabauty to
determine all rational points on interesting modular curves (as pioneered in \cite{bib-balakrishnan-dogra-muller-tuitman-vonk}), in cases
that were as far as we know previously undetermined.
\end{abstract}

\maketitle

\bibliographystyle{amsplain}
\begin{thebibliography}{normalsize}

\bibitem{bib-balakrishnan-bradshaw-kedlaya}Balakrishnan, Jennifer S., Robert W. Bradshaw, and Kiran S. Kedlaya. \textit{Explicit Coleman Integration for Hyperelliptic Curves}. In ANTS-IX 2010, LNCS 6197, pp. 16-31, 2010.
\bibitem{bib-balakrishnan-dogra-muller-tuitman-vonk}Balakrishnan, Jennifer S., Netan Dogra, J. Steffen Müller, Jan Tuitman, and Jan Vonk. \textit{Explicit Chabauty-Kim for the Split Cartan Modular Curve of Level 13}. Annals of Mathematics Vol. 189, No. 3 (May 2019), pp. 885-944.
\bibitem{bib-best-coleman-harvey}Best, Alex J. \textit{Explicit Coleman integration in larger characteristic}. Proceedings of the Thirteenth Algorithmic Number Theory Symposium, The Open Book Series, 2(1), 85-102, 2019.
\bibitem{bib-best-super}Best, Alex J. \textit{Quadratic time Coleman integration on superelliptic curves}. In preparation.
\bibitem{bib-coleman-torsion}Coleman, Robert F.  \textit{Torsion points on curves and p-adic abelian integrals.} Annals of Mathematics 121.1 (1985): 111-168.

\end{thebibliography}


\end{document}
