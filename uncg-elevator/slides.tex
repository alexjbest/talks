\documentclass[notheorems]{beamer}
%\documentclass[handout]{beamer}
%\documentclass[handout,notes=show]{beamer}


\usetheme{metropolis}
%\usecolortheme{dolphin}
% No navigation bars
\beamertemplatenavigationsymbolsempty{}

\usepackage{amsmath, amssymb, amsfonts, tikz}
\usepackage[utf8]{inputenc}
\usepackage[T1]{fontenc}
\usepackage[english]{babel}

\usepackage{amsthm}
\theoremstyle{plain}
\newtheorem{theorem}{Theorem}[section]
\newtheorem{corollary}[theorem]{Corollary}
\newtheorem{lemma}[theorem]{Lemma}
\newtheorem{algorithm}[theorem]{Algorithm}
\newtheorem{proposition}[theorem]{Proposition}
\newtheorem{claim}[theorem]{Claim}
\newtheorem{fact}[theorem]{Fact}
\newtheorem{conjecture}[theorem]{Conjecture}
%% Definition-like environments, normal text
%% Numbering is in sync with theorems, etc
\theoremstyle{definition}
\newtheorem{definition}[theorem]{Definition}
%% Remark-like environments, normal text
%% Numbering is in sync with theorems, etc
\theoremstyle{definition}
\newtheorem{remark}[theorem]{Remark}
\newtheorem{observation}[theorem]{Observation}
%% Example-like environments, normal text
%% Numbering is in sync with theorems, etc
\theoremstyle{definition}
\newtheorem{example}[theorem]{Example}
\newtheorem{question}[theorem]{Question}
\newcommand{\terminology}[1]{\textbf{#1}}

\newcommand{\NN}{\mathbf{N}}
\newcommand{\ZZ}{\mathbf{Z}}
\newcommand{\QQ}{\mathbf{Q}}
\newcommand{\RR}{\mathbf{R}}
\newcommand{\lt}{<}
\newcommand{\gt}{>}
\newcommand{\amp}{&}
\newcommand{\diff}{\mathrm{d}}

\usepackage{enumitem}



\author{Alex J. Best}
\institute{Boston University}
\date{5/28/2018}
\title{Hi! \includegraphics[width=1em]{happy.png}}


\begin{document}

\maketitle

%\begin{frame}
%\frametitle{Table of Contents}
%\tableofcontents[currentsection]
%\end{frame}

\begin{frame}{Coleman integration}
    If \(C/\RR\) is a curve, \(P,Q \in C\), \(\omega \in \Omega^1_C\) (e.g. \(\frac{x \diff x}{y}\)), we have a path integral
    \[\int_P^Q \omega\in \RR\text{.}\]
    \pause%

    What about if \(C/\QQ_p\)?

    Coleman defined
    \[\int^Q_P \omega\in \QQ_p\,\includegraphics[width=2em]{explode.png}\]
    \pause%
    a ``path'' integral, with cool properties \includegraphics[width=2em]{cool.png}.

    \pause%

    These can be explicitly computed in many cases! \includegraphics[width=2em]{keyboard.png}
\end{frame}

\begin{frame}{Applications}
    \textbf{Rational points:}
    We can sometimes find \(\omega\) so that
    \[\operatorname{Zeroes}\left( \int_{p_0}^x \omega\right) \supseteq C(\QQ)\text{ }\includegraphics[width=2em]{detective.png}\]

    \pause {\bf Heights:} \\
    Coleman-Gross introduced a height pairing on abelian varieties, it be decomposed as a sum of local terms, one of which is
    \[h_p(D_1,D_2) = \int_{D_2} \omega_{D_1}\,\includegraphics[width=2em]{pair.png}\]

    \pause {\bf \(p\)-adic BSD:} \\
    Using the above height pairing one can define a \(p\)-adic regulator so that for a modular abelian variety \(A/\QQ\) conjecturally
    \[\mathcal L^* (A,0)  = \epsilon_p(A)\frac{|\underline{\mathrm{III}}(A/\QQ)|\operatorname{Reg}_{\gamma}(A/\QQ)\prod_v c_v}{|A(\QQ)_{\text{tors}}||A^ \vee(\QQ)_{\text{tors}}|}\text{ } \includegraphics[width=2em]{dizzy.png}\]

\end{frame}

\end{document}
