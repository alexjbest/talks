%\documentclass[notheorems]{beamer}
%\documentclass[handout]{beamer}
%\documentclass[handout,notes=show]{beamer}

\usetheme{metropolis}
%\usecolortheme{dolphin}
% No navigation bars
\beamertemplatenavigationsymbolsempty

\usepackage{amsmath, amssymb, amsfonts, tikz}
\usepackage[utf8]{inputenc}
\usepackage[T1]{fontenc}
\usepackage[english]{babel}
\usepackage{tabularx}
\usepackage{mathtools}
\usepackage{xmpmulti}


\usepackage{amsthm}
\theoremstyle{plain}
\newtheorem{theorem}{Theorem}[section]
\newtheorem{corollary}[theorem]{Corollary}
\newtheorem{lemma}[theorem]{Lemma}
\newtheorem{algorithm}[theorem]{Algorithm}
\newtheorem{proposition}[theorem]{Proposition}
\newtheorem{claim}[theorem]{Claim}
\newtheorem{fact}[theorem]{Fact}
\newtheorem{conjecture}[theorem]{Conjecture}
%% Definition-like environments, normal text
%% Numbering is in sync with theorems, etc
\theoremstyle{definition}
\newtheorem{definition}[theorem]{Definition}
%% Remark-like environments, normal text
%% Numbering is in sync with theorems, etc
\theoremstyle{definition}
\newtheorem{remark}[theorem]{Remark}
\newtheorem{observation}[theorem]{Observation}
%% Example-like environments, normal text
%% Numbering is in sync with theorems, etc
\theoremstyle{definition}
\newtheorem{example}[theorem]{Example}
\newtheorem{question}[theorem]{Question}
\newcommand{\terminology}[1]{\textbf{#1}}

\newcommand{\NN}{\mathbf{N}}
\newcommand{\ZZ}{\mathbf{Z}}
\newcommand{\QQ}{\mathbf Q}
\newcommand{\CC}{\mathbf C}
\newcommand{\RR}{\mathbf R}
\newcommand{\FF}{\mathbf F}
\newcommand{\lt}{<}
\newcommand{\gt}{>}
\newcommand{\amp}{&}
\newcommand{\diff}{\mathop{}\!\mathrm{d}}
\newcommand{\ints}{\mathcal{O}}
\newcommand{\ideal}[1]{\mathfrak{#1}}
\usepackage{mathrsfs}\usepackage{cancel}
\newcommand{\Gal}[2]{\operatorname{Gal}(#1/#2)}
\newcommand{\absgal}[1]{\operatorname{Gal}(\overline{#1}/#1)}
\DeclareMathOperator{\USp}{USp}
\DeclareMathOperator{\Spec}{Spec}

\newcommand{\sheaf}[1]{\operatorname{\mathcal{#1}}}
\newcommand{\inv}{^{-1}}
\DeclareMathOperator{\norm}{Nm}
\DeclareMathOperator{\ord}{ord}
\DeclareMathOperator{\divisor}{div}
\DeclareMathOperator{\PP}{\mathbf{P}}
\DeclareMathOperator{\Hom}{Hom}
\DeclareMathOperator{\Mat}{Mat}
\DeclareMathOperator{\End}{End}

\newcommand{\lb}{[}
\newcommand{\rb}{]}

\author{Alex J. Best}
\institute{AMS Graduate Student Conference, Brown University}
\date{16/4/2019}
\title{Zeta functions and $p$-adic integrals; computations and applications}

\begin{document}

\begin{frame}
  \titlepage

  \note[item]{Thank the audience for being awake.}
\end{frame}

%\begin{frame}
%\frametitle{Table of Contents}
%\tableofcontents[currentsection]
%\end{frame}


% Overview :
% - Zeta functions, history
% - applications
%   - finding maps between curves
%   - global l-functions?
%   - point counts on curve and jac
%   - katz? n-torsion theorem
% - how to compute?
% - Kedlaya, variants
% - Coleman integrals
% - how to compute?
% - BBK, variants
% - applications
%   - torsion points
%   - rational points
%     - classical chab
%     - cursed curve (iterated)
%     - BBTV

\begin{frame}{Where are we going?}
    \begin{quotation}... Hilbert often interrupted me... he kept interrupting frequently--
        finally I could not speak any more at all -- and he said that from the start he
        did not even listen since he had the impression that everything was trivial
        ---E. Artin
    \end{quotation}

    \pause
    \textbf{Please ask questions!}
    \pause

    \textbf{Plan:}
    \begin{itemize}
        \item
            Zeta functions:
            \begin{itemize}
                \item What are they?
                \item Why calculate them?
                \item How do you find them?
            \end{itemize}\pause
        \item
            Coleman integrals:
            \begin{itemize}
                \item What are they?
                \item Why calculate them?
                \item How do you find them?
            \end{itemize}
    \end{itemize}
\end{frame}


\begin{frame}{Curves and their points}
    Let $C$ be a (smooth, projective) curve over $\FF_q$, a finite field with $q$ elements.
    \pause

    As $\FF_q$ is finite $C(\FF_q)$ is finite, moreover $C(\FF_{q^n})$ is finite for all $n$, what are the values for different $n$?  \pause
    \begin{example}
        If $C  = \PP^1/\FF_p$ then we have $C (\FF_q) = \FF_q \cup\{\infty\}$ so
        \[\#C(\FF_{p^n}) = p^n + 1\text.\]
    \end{example}
\end{frame}

\begin{frame}{An elliptic curve}
    \begin{example}
        If $E\colon y^2 = x^3 - 1/\FF_5$ then
        \begin{table}
            \centering
            \resizebox{\columnwidth}{!}{%

                \begin{tabular}{r|cccccccc}
                    \(n\)&\(1\)&\(2\)&\(3\)&\(4\)&\(5\)&\(6\)&\(7\)&\(8\)\tabularnewline[0pt]
                    \hline
                    \(\#E(\FF_{5^n})\)&\(6\)&\(36\)&\(126\)&\(576\)&\(3126\)&\(15876\)&\(78126\)&\(389376\)\pause\tabularnewline[0pt]
                    \(5^n\)&\(5\)&\(25\)&\(125\)&\(625\)&\(3125\)&\(15625\)&\(78125\)&\(390625\)\pause\tabularnewline[0pt]
                    \(\#E(\FF_{5^n}) - 5^n - 1\)&\(0\)&\(10\)&\(0\)&\(-50\)&\(0\)&\(250\)&\(0\)&\(-1250\)
                \end{tabular}
            }
        \end{table}

        \only<4->{We need a formula that is $0$ for odd $n$ and $-2\cdot (-5)^{n/2}$ for even $n$:
            \[\#E(\FF_{5^n}) = 5^n + 1 - \left(\sqrt{-5}^n + (-\sqrt{-5})^n \right)\]
        }

        \only<5->{It initially seemed like we had an infinite amount of data here: $\#E(\FF_{5^n})$ for all $n \in \NN$. But we don't!}
    \end{example}


\end{frame}



\begin{frame}{The Weil polynomial}
    Rephrased: we have a polynomial
    \[L_E = t^2 + 5\]
    so that
    \[\#E(\FF_{5^n}) = 5^n + 1 - \sum_{\text{roots }\alpha_i \text{ of }L_E} \alpha_i^n\]
    how general a phenomenon is this?\pause

    \begin{theorem}[Schmidt?, Weil?]
        Let $C / \FF_q$ be a curve, there exists a monic $L_C(t) \in \ZZ\lb t \rb$ of degree $2 \cdot \operatorname{genus}(C)$. Whose roots $\alpha_i$ come in complex conjugate pairs with $|\alpha_i| = q^{1/2}$ and
        \[\#C(\FF_{q^n}) = q^n + 1 - \sum_{\text{roots }\alpha_i \text{ of }L_C} \alpha_i^n\]
    \end{theorem}


\end{frame}

\begin{frame}{The zeta function}
    The condition on the roots means $\alpha_i \overline \alpha_i = q$ so we may write $L_C(t) = q^g\prod_i (1-\frac{\alpha_i }{q}t)$ then
    \[\log(L_C(t)/q^g) = -\sum_i \sum_{n=1}^\infty \frac{\alpha_i^n t^n}{q^n n}  = \sum_{n=1}^\infty -\left(\sum_i \alpha_i^n\right) \frac{t^n}{q^n n}\]
    so \(\log(L_C(qt)/q^g)\) almost knows the point counts\pause, if we define:

    \begin{definition}
        The (Hasse-Weil) zeta function of $C/\FF_q$ is
        \[ Z(C, t) \coloneqq \exp\left(\sum_{i=1}^\infty \#C(\FF_{q^i}) \frac{t^i}{i}\right) \]
    \pause
    And we have that \[Z(C,t) = \frac{q^{-g}L_C(qt)}{(1-t)(1-qt)}\text.\]
    \end{definition}
\end{frame}

\begin{frame}{Why bother?}
    \textbf{Reverse engineering:}
    Find point counts!

    If we have a way to find the zeta function we can get the point counts in a more sophisticated way.\pause

    In fact if \(J =\operatorname{Jac}(C)\) the Jacobian (i.e.\ the class group of \(C\))
    \[ L_C(1) = \# J(\FF_q)\]

    We can tell a lot about the Jacobian from this number!\pause

    \begin{example}[A completely random example, I promise]
        \[C \colon y^2 = x^5 + 6x^2 + x + 3/\FF_{43}\]
        \[L_C(t) = t^4 + 9t^3 + 64t^2 + 387t + 1849\]
        \[ \implies\#J(\FF_{43}) = L_C(1) = 2310 = 2 \cdot 3 \cdot 5 \cdot 7 \cdot 11\]
        so \(J(\FF_{43}) = C_{2 \cdot 3 \cdot 5 \cdot 7 \cdot 11}\).  \pause
        So never use this curve for cryptography!!
    \end{example}
\end{frame}

\begin{frame}{Distributional questions - Sato-Tate}
    Let $C/\QQ$ be a genus $g$ curve. We can reduce mod $p$ for all primes $p$ of good reduction, get a polynomial $L_{C_{\FF_p}}(t)$ for all these $p$. \pause
    If we \emph{normalise} to have all roots of complex norm 1 we get
    \[\widetilde L_{C_{\FF_p}}(t) = L_{C_{\FF_p}}\left(\sqrt{p} t\right)\text,\]
    a unitary symplectic polynomial, i.e.\ the characteristic polynomial of a unitary symplectic matrix.\pause

    So we get a map
    \[\text{good primes} \to \operatorname{Conj}(\USp(2g))\]
    the RHS has a Haar measure coming from $\USp(2g)$\pause

    How is the image distributed as $p\to \infty$?
\end{frame}

\begin{frame}{Genus 1}
    %\transduration<0-30>{0}
    \[y^2 = x^3 + x+ 1 \qquad y^2 =x^3 + 1\]
    \multiinclude[<+->][format=png, graphics={width=\textwidth}]{img/g1}
    Pictures due to Drew Sutherland.
    \pause
    Left is a generic elliptic curve, the right has CM (over $\QQ$).
    \pause
    By computing enough zeta functions we can \emph{see} the endomorphism algebra of our curve.
\end{frame}

\begin{frame}{Genus 2}
    %\transduration<0-30>{0}
    \[y^2 = x^5 - x + 1 \qquad y^2 =x^6 + 2\]
    \multiinclude[<+->][format=png, graphics={width=\textwidth}]{img/g2}
    Pictures due to Drew Sutherland.
    \pause
    Left is a generic genus 2 curve, the right has $\End(\operatorname{Jac}(C)_{\overline{\QQ}})\otimes \RR = \Mat_2(\CC)$.
    \pause
    After the work of Fit\'e-Kedlaya-Rotger-Sutherland we can recognise these distributions and guess the structure of the Jacobian, the right one should be square of a CM elliptic curve.
\end{frame}

\begin{frame}{Relations}
    Let
    \[ C_1 \colon y^2 + y = x^3 + x/\FF_2,\, C_2 \colon y^2 + y = x^5 + x/\FF_2\]\pause
    then
    \[ L_{C_1}(t) = t^2 + 2t + 2 ,\,L_{C_2} = ( t^2 + 2t + 2)(t^2 + 2)\]
    what does this tell us?\pause

    \begin{theorem}[Kleiman, Serre]
        If there is a morphism of curves \(C\to D\) over \(\FF_q\) then
        \[L_D(t) | L_C(t)\]
    \end{theorem} \pause

    In our example we have a map
    \[(x,y) \mapsto (x^2 + x, y + x^3 + x^2)\text.\]

\end{frame}

\begin{frame}{Relations again}
    The converse is false!
    \[ D_1 \colon y^2 + xy = x^5 + x/\FF_2,\, D_2 \colon y^2 + xy = x^7 + x/\FF_2\]\pause
    where
    \[ L_{D_1}(t) = t^4 + t^3 + 2t + 4 ,\,L_{D_2} = ( t^4 + t^3 + 2t + 4 )(t^2 + 2)\]
    but no map exists!
\end{frame}


\begin{frame}{How to compute?}
    \textbf{Reverse reverse engineering:}
    Count points for a few $n$ ($n\le g$ is sufficient), recover $L_C(t)$.  \pause
    This can take a long time! \pause

    \textbf{$p$-adic cohomology:}
    A method due to Kedlaya relates $L_C(t)$ to $p$-adic cohomology.
    $L_C(t)$ is the characteristic polynomial of ``Frobenius'' acting on ``$H^1_{MW}(\tilde C)$''.
    If we can compute this action (as a matrix) we win!
    \pause

    \textbf{Average time:}
    Harvey-Sutherland have an approach to compute $L_{C_{\FF_p}}(t)$ for a curve over $\QQ$ for all $p \lt N$ at once!
    This works out faster on average.
\end{frame}

\begin{frame}{Monsky-Washnitzer cohomology in general}
    Let $C /\FF_q$ be an (odd) hyperelliptic curve. \pause

    First choose a lift $\tilde C/\ZZ_q$ and an affine open $U = \Spec(A) \subseteq C$.
    And a lift of the $q$-power Frobenius on \(\overline A= A/pA\) to \(\phi\colon A^\dagger \to A^\dagger\).

    Now the weak completion $A^\dagger$ is the set of $p$-adic power series on $U$ that $p$-adically overconverge.

    We have differentials \(\Omega_{A^\dagger}^1\) and a derivative i\(\diff \colon A^\dagger \to \Omega^1_{A^\dagger}\)
    %We will say that \(f\) is a \terminology{primitive} of the exact differential \(\diff f\).
    \[H^1_{\mathrm{MW}}(\overline A) = \Omega^1_{A^\dagger}\otimes \QQ_p/\diff(A^\dagger\otimes \QQ_p)\]


\end{frame}

\begin{frame}{Monsky-Washnitzer cohomology (for hyperelliptic curves)}
    Let $C \colon y^2 = \overline Q(x)/\FF_q$ be an (odd) hyperelliptic curve. \pause

    First choose a lift $\tilde C \colon y^2 = Q(x)/\ZZ_q$. \pause

    The affine coordinate ring of the punctured curve is \[A = \ZZ_p[x,y,y^{-1}]/(y^2 - Q(x))\]\pause
    \[A^\dagger =  \left\{ \sum_{i = -\infty}^\infty R_i(x)y^{-i} : R_i \in \ZZ_p[x]_{\deg \le 2g}\text{ where }\liminf_{|i| \to \infty} v_p(R_i)/|i| \gt 0\right\}\]\pause

    The \(q\)-power Frobenius on \(A/pA\)  can be lifted to \(\phi\colon A^\dagger \to A^\dagger\)
    \begin{align*}
        x \amp\mapsto x^p\\
        y \amp\mapsto y^{-p}\sum_{k=0}^{\infty}\binom{-1/2}{k}(\phi(Q(x)) - Q(x)^p)^k /y^{2pk}\text.
    \end{align*}
\end{frame}

\begin{frame}{Monsky-Washnitzer cohomology (for hyperelliptic curves)}

    \[ \Omega _{A^\dagger} = A^\dagger \diff x \oplus A^\dagger \diff y/(2y\diff y - Q'(x)\diff x))\]


    \begin{align*}
        \diff \colon A^\dagger \amp\to \Omega^1_{A^\dagger}\\ \sum_{i=-\infty}^\infty \frac{R_i(x)}{y^i} \amp\mapsto \sum_{i=-\infty}^\infty R'_i(x) y^{-i}\diff x-R_i(x) iy^{-i-1} \diff y\text{.}\label{eqn-exterior-d}
    \end{align*}

    %The action of Frobenius and of the hyperelliptic involution can be extended to \(\Omega^1_{A^\dagger}\) and to \(H^1_{\mathrm{MW}}(\overline A)\) and the actions of \(\phi\) and \(\iota\) commute. In particular we have an eigenspace decomposition of all of these spaces under \(\iota\) into \terminology{even} and \terminology{odd} parts; the odd part will be denoted with a \(-\) superscript. Let  \(A_\mathrm{loc}(X)\) denote the \(\QQ_p\)-valued functions on \(X(\QQ_p)\) which are given by a power series on each residue disk.%
\end{frame}

\begin{frame}{Reductions in cohomology}
    \(\{\omega_i  = x^i\diff x/y\}_{i=1,\ldots, 2g}\) are a basis for $H^1_{MW}(C)$ and for each $i$
    we get an expansion
    \[\phi^* \omega_i \equiv\sum_{j=0}^{N-1} \sum_{r=0}^{(2g+1)j} B_{j,r} x^{p(i+r+1) - 1}y^{-p(2j+1) + 1} \frac{\diff x}{2y}\pmod{p^N}\]\pause
    We need to write this in the form
    \[\phi^* \omega_i \equiv \sum_{j=1}^{2g} a_{ij} \omega_j  - \diff(f_i)\pmod{p^N}\]\pause
    to do this we iteratively use relations like
    \begin{align*}
        \diff(x^s y^{-2t + 1}) = \amp\left(2s - (2t - 1 ) (2g+1)  \right)x^{2g+1} x^{s-1} y^{-2t}\frac{\diff x}{2y} \\
    \amp+ \left(2sP(x) - (2t - 1 )xP'(x) \right) x^{s-1} y^{-2t}\frac{\diff x}{2y}\text{.}\end{align*}
    to reduce the exponents of monomials appearing in the expansion.
\end{frame}

\begin{frame}{Reductions in cohomology}
    We  end up with
    \[\phi^* \omega_i \equiv \sum_{j=1}^{2g} a_{ij} \omega_j  - \diff(f_i)\pmod{p^N}\]\pause
    The $L$-polynomial is then the characteristic polynomial of the matrix $F = (a_{ij})_{i,j}$.
\end{frame}

\begin{frame}{Interlude: Computing things quickly - a silly example}
    Suppose we want to evaluate $N!$ for $N$ large, how many ring operations does this take? \pause
    Naively: $N$ operations, but we can break up the product into chunks by 
    %\[ N ! = 1 \cdot 2 \cdot \cdots \cdot \left\lfloor N/2\right\rfloor \cdot (\left\lfloor N/2\right\rfloor   + 1)\cdot\cdots \cdot (N-1) \cdot N\]
    dividing into products of length $\sqrt[4]{N}$ (so $\sqrt[4]{N}^3$ subproducts in total)
    \[ N! = P(0)\cdot P(\sqrt{N})\cdot P(2\sqrt{N})\cdots P((\sqrt{N} - 1)\sqrt N) \]
    where
    \[P(x) = (x + 1)(x+2) \cdots (x+\sqrt[4]{N})\]\pause
    once we compute $\sqrt[4]{N}$ of these $P(i)$ (for $i = 0, \ldots, \sqrt[4]{N}$) in $\sqrt{N}$ steps we have a degree $\sqrt[4]{N}$ polynomial evaluated at $\sqrt[4]{N}$ points. If you know a (monic) degree $n$ polynomial at $n$ points, you know the polynomial! $\leadsto$ interpolate to find other values
\end{frame}

\begin{frame}{Interlude: Computing things quickly - Fancy version}
    In general if we have $M(t)\in \Mat_{n\times n}(R[t])$ a matrix with linear polynomials as coefficients.
    We can evaluate lots of products
    \begin{align*}\amp M(0) M(1) \cdots M(k-1),\\\amp M(k) M(k+1) \cdots M(2k-1),\\\amp\vdots\\\amp M((m-1)k) M((m-1)k+1) \cdots M(mk - 1)\end{align*}\pause
    quickly in practice! (Bostan-Gaudry-Schost,Harvey)\pause

    Using this we can reduce quickly and compute $L_C(t)$ in time roughly $\sqrt p$ (Harvey).

    With Arul, Costa, Magner, Triantafillou we can do this for general cyclic covers $y^a = f(x)$.
\end{frame}

\begin{frame}{Part II - Coleman integrals}
    Take \(C/\ZZ_p\) a genus \(g\) curve and \(p\) an odd prime.
    \begin{theorem}[{Coleman}]
        There is a \(\QQ_p\)-linear map \(\int_b^x\colon \Omega_{A^\dagger}^1\otimes \QQ_p \to A_\mathrm{loc} (X)\) for which:
        \begin{align*}
            \diff \circ \int_b^x \amp= (\mathrm{id}\colon \Omega_{A^\dagger}^1\otimes \QQ_p \to \Omega_{\text{loc}}^1)\,\quad\text{``FTC''}\\
            \int_b^x\circ\diff \amp = (\mathrm{id}\colon A^\dagger \hookrightarrow A_{\mathrm{loc}})\\
        \int_b^x \phi^*\omega \amp= \phi^*\int_b^x \omega\,\quad\text{``Frobenius equivariance''}
        \end{align*}
    \end{theorem}\pause
    Locally we can integrate power series formally.\pause

    To integrate between far away points we use Frobenius equivariance.
\end{frame}

\begin{frame}{Frobenius equivariance}
    Switch to an odd hyperelliptic curve now, some manipulation with the set of all $\int_P^\infty\omega_i$ gives:
    \begin{equation*}
        \left(\begin{smallmatrix} \vdots \\ \int_{P}^\infty \omega_i \\\vdots \end{smallmatrix}\right) = (F - I)^{-1} \left(\begin{smallmatrix}\vdots \\ f_i(P) \\ \vdots \end{smallmatrix}\right)
    \end{equation*}
    where from earlier
    \[\phi^* \omega_i \equiv \sum_{j=1}^{2g} M_{ij} \omega_j  - \diff f_i\]
\end{frame}

\begin{frame}{Effective Chabauty}
    One consequence of Coleman's work we saw earlier is
    \begin{theorem}[Coleman's effective Chabauty]
        Let $C/\QQ$ be a curve of genus $g$. If $\operatorname{rank}J(C)(\QQ)\lt g$ and $p\gt 2$ is a prime of good reduction for $C$ then
        \[\#C(\QQ) \le \# C_p(\FF_p) + 2g-2\text.\]
    \end{theorem}
\end{frame}

\begin{frame}{Explicit Chabauty}
    Given an individual curve we can often compute $X(\QQ)$ by explicitly evaluating enough of these integrals.\pause

    More generally via non-abelian Chabauty we can approach more curves, \pause this requires computing iterated Coleman integrals.

    \[X_s(13)\] \pause

    or

    \[X_0(67)^+\text{ and friends}?\]

\end{frame}

\begin{frame}{A fun converse}
    Thinking about effective Chabauty backwards: if we have a lot of $\QQ$-points and few $\FF_p$ points, the Jacobian must have large rank!

    \begin{example}
        To force a curve to have many $\QQ$ points and few $\FF_7$ points, let
        \[ C\colon y^2 = x(x-7)(x-14)(x+ 7)(x+14) + 1\]\pause
        this has a bunch of rational points $(7n, \pm 1)$ for $n =-2,-1,0,1,2$ (and $\infty$ so $\ge11$ in all), but these give the same $\FF_7$ points $(0,\pm1)$.
        In fact
        \(\#C(\FF_7) = 8\)
        so we fail the Coleman bound as
        \[ 11 \le 8 + 2g - 2= 10\]
        so we must have $\operatorname{rank} \operatorname{Jac}(C)(\QQ) \ge g =2$.
        (Magma tells me $\operatorname{rank} \operatorname{Jac}(C)(\QQ) = 5$ in fact!)
    \end{example}
\end{frame}

\begin{frame}{More generally}
    In fact Coleman showed:
    \begin{corollary}[Coleman]
        Let $k\in \ZZ$, $p\nmid k$ prime and $f(x)/\ZZ$ monic with $f(x) \equiv x^k \pmod p$ and $\lfloor (k+1)/2 \rfloor$ roots over $\ZZ$ then the rank of the Jacobian of
        \[y^2  = f(x) + 1\]
        is at least the genus (which is $\lfloor (k-1)/2\rfloor$)
    \end{corollary}\pause

    The proof is a little more serious than our example above, it shows that the points $(\alpha_i, 1)$ where $\alpha_i$ are roots of $f$ are actually linearly independent in the Jacobian.
\end{frame}

\begin{frame}{Fast Coleman}
    Recall to compute a Coleman integral we need to find
    \begin{equation*}
        F, \{f_i(P)\}_i
    \end{equation*}\pause
    we can coerce the evaluation of $f_i(P)$ into a linear recurrence and apply Bostan-Gaudry-Schost!
\end{frame}

\begin{frame}{Generalities}
    Coleman integration can be more general, Coleman de Shallit define:
    \[
        r_C \colon K_2(\overline k(C))  \to \Hom (H^0(C, \Omega_{C/\overline k}^1), \overline k)\text{.}
    \]
    \begin{equation*}
        r(f,g)(\omega) = -\int_{(g)} \log(f) \in \overline k
    \end{equation*}

\end{frame}

\begin{frame}{Where next?}
    \begin{itemize}
        \item Coleman integration quickly on general curves
        \item Coleman integration for many primes at once?
        \item Distribution?
    \end{itemize}
\end{frame}
%\begin{frame}{Stoll's sharpening}
%    Michael Stoll has shown that in fact assuming that $\operatorname{rank}(J) = r\lt g$ and that $p \gt 2r +2$ we have
%    \[ \#C(\QQ) \le \#C(\FF_p) + 2r\text.\]\pause

%    We can reverse this too!
%    \begin{example}
%        \[ C\colon y^2 = \prod_{i =0}^6 (x - 13i)+ 1\]\pause
%        has $\#C(\FF_{13}) = 14$ and at least $15$ rational points.

%    \end{example}

%\end{frame}


\end{document}
