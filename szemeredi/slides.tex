\documentclass[oneside,12pt,]{article}
\usepackage[T1]{fontenc}
\usepackage[utf8x]{inputenc}
\usepackage{geometry}
\usepackage[english]{babel}
\usepackage{lmodern}
\usepackage[skip=\baselineskip,indent=0pt]{parskip}

\usepackage[bookmarksopen=true,bookmarks=true]{hyperref}
\geometry{paperwidth=170mm, paperheight=4383pt, left=10pt, top=20pt, textwidth=340pt, marginparsep=20pt, marginparwidth=100pt, textheight=4263pt, footskip=40pt}

\usepackage[explicit, newparttoc, pagestyles]{titlesec}
\usepackage{titletoc}
\usepackage{graphicx}
\input{insbox}%%%%%%%%%%%%%% TeX macro,

\graphicspath{{img/}}


\usepackage{float}

\usepackage{mathrsfs}
\usepackage{tikz}
%\usepackage{tikz-3dplot}
%%\tdplotsetmaincoords{60}{110}
\usepackage{wrapfig}

\usepackage{amsmath,mathtools}
\usepackage{amscd}
\usepackage{amssymb}
\usepackage{amsthm}
\newtheorem{theorem}{Theorem}
\newtheorem{lemma}[theorem]{Lemma}
\newtheorem{definition}[theorem]{Definition}

\usepackage{newpxtext,newpxmath}
\setcounter{secnumdepth}{0} % sections are level 1

\DeclarePairedDelimiter\ceil{\lceil}{\rceil}
\DeclarePairedDelimiter\floor{\lfloor}{\rfloor}
\DeclareMathOperator{\Spec}{Spec}
\DeclareMathOperator{\Proj}{Proj}
\DeclareMathOperator{\Spf}{Spf}
\DeclareMathOperator{\Spv}{Spv}
\DeclareMathOperator{\Spa}{Spa}
\DeclareMathOperator{\Spm}{Spm}
\DeclareMathOperator{\specialisation}{sp}
\DeclareMathOperator{\Max}{Max}
\newcommand{\Gal}[2]{\operatorname{Gal}(#1/#2)}
\newcommand{\absgal}[1]{\operatorname{Gal}(\overline{#1}/#1)}
\newcommand{\sepgal}[1]{\operatorname{Gal}(#1^\sep/#1)}
\DeclareMathOperator{\Ind}{Ind}
\DeclareMathOperator{\Res}{Res}
\DeclareMathOperator{\res}{res}

\newcommand{\diff}{\mathop{}\!\mathrm{d}}
\newcommand{\cinf}{C^\infty}
\newcommand{\inv}{^{-1}}
\newcommand{\units}{^{\times}}

\newcommand{\legendre}[2]{\left(\frac{#1}{#2}\right)}
\newcommand{\pair}[2]{\left\langle #1, #2 \right\rangle}
\newcommand{\lb}{[}
\newcommand{\rb}{]}
\newcommand{\powerseries}[2]{#1[[#2]]}

\DeclareMathOperator{\power}{\mathcal{P}}
\DeclareMathOperator{\aff}{\mathbf{A}}
\DeclareMathOperator{\PP}{\mathbf{P}}
\DeclareMathOperator{\norm}{Norm}
\DeclareMathOperator{\trace}{Tr}
\DeclareMathOperator{\Fr}{Fr}
\DeclareMathOperator{\Frob}{Frob}
\DeclareMathOperator{\NS}{NS}
\DeclareMathOperator{\Der}{Der}
\DeclareMathOperator{\Aut}{Aut}
\DeclareMathOperator{\Out}{Out}
\DeclareMathOperator{\Inn}{Inn}
\DeclareMathOperator{\vf}{\mathcal{V}}
\DeclareMathOperator{\krulldim}{krulldim}
\DeclareMathOperator{\trdeg}{trdeg}
\DeclareMathOperator{\Frac}{Frac}
\DeclareMathOperator{\Prob}{Prob}

\DeclareMathOperator{\ch}{ch}
\newcommand{\lt}{<}
\newcommand{\gt}{>}
\newcommand{\amp}{&}

\newcommand{\NN}{\mathbf{N}}
\newcommand{\ZZ}{\mathbf{Z}}
\newcommand{\QQ}{\mathbf{Q}}
\newcommand{\RR}{\mathbf{R}}
\newcommand{\CC}{\mathbf{C}}
\newcommand{\HH}{\mathbf{H}}
\newcommand{\FF}{\mathbf{F}}
\newcommand{\GG}{\mathbf{G}}
\newcommand{\ints}{\mathcal{O}}
\newcommand{\adeles}{\mathbf{A}}

\title{Szemerédi's regularity lemma}
\author{Alex J. Best}
\date{31/6/23}
\begin{document}
\maketitle
\emph{Recap:} Studying Manin's conjecture and (equi)-distribution of rational points on a Fano variety. Another instance of pseudo-randomness emerging with scale is in extremal combinatorics / graph theory.

\emph{Tao (paraphrased):} ``The various proofs of Szemerédi's theorem and related theorems and proofs using measure theory, ergodic theory, graph theory, hypergraph theory, probability theory, information theory, and Fourier analysis share a number of common features and serve as a ``Rosetta stone'' for connecting these fields, notably they often use dichotomy between randomness and structure''.

\emph{This time:} Give more detail on Szemerédi's regularity lemma (SRL) and its proof, and the reduction to Roth's theorem / Szemerédi's theorem.


\section{The statement}

\emph{Slogan:} 
The vertices of a sufficiently large graph can be partitioned into a fixed number of subsets in a way that the interactions between each behave pseudorandomly.

There are several variants of SRL, some place more restrictions on the partition (and so appear at first sight stronger).

\begin{definition}

    Let $G= (V, E)$ be a graph and $A, B \subseteq V$ two disjoint sets.
    The density of edges between $A$ and $B$ is
    \InsertBoxR{0}{\begin{minipage}{0.35\linewidth}\centering
            \hspace{1000mm}
    \end{minipage}%
}[0]
    $$
    d(A, B)=\frac{e(A, B)}{|A||B|}
    $$ %Note that the density is always a positive value between 0 and 1 .
    We call a pair $(A, B)$ $\epsilon$-regular (or uniform) for a given $\epsilon>0$, if for all $X \subseteq A, Y \subseteq B$ where $|X| \geq \epsilon|A|$ and $|Y| \geq \epsilon|B|$, we have:
    $$
    |d(X, Y)-d(A, B)| \leq \epsilon
    $$
\end{definition}

\begin{definition}
    A partition $V_1, \ldots, V_k$ of the vertices of a graph is said to be an equipartition if it is as balanced as possible, i.e.
$$\max\{||V_k| - |V_j||\}\le 1\text{ or }\floor*{\frac{|V|}k} \le |V_j|\le\ceil*{\frac{|V|}k}$$
\end{definition}

\begin{definition}
    A partition $V_1, \ldots, V_k$ of the vertices of a graph is said to be $\epsilon$-regular if most pairs $(V_i,V_j)$ are $\epsilon$-regular, in the sense that at most
    $$\epsilon \binom k 2\text{ are not }\epsilon\text{-regular}$$
\end{definition}


\begin{theorem}[Szemerédi] Let $\varepsilon>0$, and let $l$ be a natural number. Then there exists an integer $L$ such that every graph $G$ with $|G| \geq l$ has an $\varepsilon$-uniform equipartition into $m$ parts for some $m$ such that $l \leq m \leq L$.
\end{theorem}

Note that $L$ does not depend on $G$ or even the size of $G$, so that for large enough graphs we can consider this a partition into a small number of pieces relative to the size of the graph.


\section{Applications}

\begin{theorem}[Roth]
    A subset of the natural numbers with positive upper density contains a 3-term arithmetic progression.
\end{theorem}


Szemerédi was motivated by generalizing this result and proved:

\begin{theorem}[Szemerédi]
    A subset of the natural numbers with positive upper density contains a 
     $k$-term arithmetic progression for any $k$.
\end{theorem}

\section{Method}

\begin{theorem}[Roth']
    For every $\delta>0$, there exists an $n_0$ such that for any $n \geq n_0$ and any subset $A \subseteq\{1, \ldots, n\}$ satisfying $|A| \geq \delta n$, there are distinct elements $a, b, c \in A$ such that $a+c=2 b$.
\end{theorem}


To prove this we instead let
$$B = \{(x,y) : x - y \in A\} \subseteq \{1, \ldots, 2n\}^2$$

\begin{definition}[Corners] A \emph{corner} in a set $B \subseteq \{1, \ldots, n\}^2$ is a triple of the form $(x,y), (x+h, y), (x, y+h) \in B$, $0 < h$ (anticorner if $h < 0$).
\end{definition}
So, given a corner in $B$ we get a 3 term AP $(a,b,c)$ in $A$ from $(x-y, x+h-y, x - y - h)$.

\begin{theorem}[Corners theorem] For every $\delta>0$, there exists an $n_0$ such that for any $n \geq n_0$ and any subset $B \subseteq\{1, \ldots, n\}^2$ satisfying $|B| \geq \delta n^2$, there is a corner in $B$.
\end{theorem}

Reduce to the weak corners theorem (as above but allowing anticorners).

Then construct a tripartite graph where the triangles correspond to (anti or trivial)-corners of $B$ and so that all triangles are edge disjoint.

The construction is to have vertices for horizontal, vertical and diagonal lines in $\{1, \ldots, 2n\}$ and put an edge when two such lines meet at a point of $B$.

There are at least $\delta n^2$ triangles in this graph from trivials alone so to remove all of them we must remove at least $\delta n^2$ edges.


\begin{theorem}[Triangle Removal Lemma] For all $1\ge \delta >0$, there exists $\epsilon >0$ such that any graph on $n$ vertices with less than or equal to $\epsilon  n^3$ triangles can be made triangle-free by removing at most $\delta  n^2$ edges.

    (If there are not too many triangles you can remove a small number of edges to remove all triangles.)
\end{theorem}

(this is a special case of the more general \emph{Graph removal lemma} and the \emph{hypergraph removal lemma} (used for the full Szemerédi lemma on $k$-term APs).)

Now the above graph must have at least $\epsilon n^3$ triangles as the triangle removal lemma does not apply.

So we have at least
$$\epsilon   n^3 - \delta n^2$$
nontrivial triangles, so pick $n$ such that $\epsilon n > \delta $ and we are done.


Roth's theorem can be proved directly from the TRL, but the \emph{Corners theorem} is in some sense a stronger version of Roth.

%\section{Proof of Triangle removal lemma from SRL}

We can prove the TRL from the so called triangle counting lemma, from SRL (though other proofs are available).

\section{Proof of SRL}
The proof idea is fairly straightforward in the outline:

\begin{itemize}
    \item We define a (bounded above) quantity called \emph{energy} of an equipartition, that unless the equipartition is $\epsilon $-regular can be increased by at least a fixed positive amount (via some modification of the equipartition), without adding too many sets to the equipartition.

    \item We start with a trivial equipartition and may inductively apply this process which must eventually stop (after a number of steps bounded independently of the input graph) at which point we have proved the lemma.
\end{itemize}


The details of this are quite involved however!

\begin{definition}
    The \emph{energy} of a partition $P= \{V_1, \ldots, V_k\}$ is
    $$0 \le q_G(P)=\frac{1}{k^2} \sum_{1 \leq i<j \leq k} d_G\left(V_i, V_j\right)^2 \le  1$$
\end{definition}

\begin{lemma}
    Let $G$ be a graph of order $n$ with an equipartition $V=\bigcup_{i=0}^k C_i$
    $$
    \left|C_1\right|=\left|C_2\right|=\cdots=\left|C_k\right|=c \geq 2^{3 k+1} .
    $$
    Suppose that the partition $\mathcal{P}=\left(C_i\right)_{i=0}^k$ is not $\varepsilon$-uniform, where $0<\varepsilon<\frac{1}{2}$ and $2^{-k} \leq \varepsilon^5 / 8$. Then there is an equitable partition $\mathcal{P}^{\prime}=\left(C_i^{\prime}\right)_{i=0}^{\ell}$ with $\ell = k (4^k - 2^{k-1})$ such that
    $$
    q\left(\mathcal{P}^{\prime}\right) \geq q(\mathcal{P})+\frac{\varepsilon^5}{2}
    $$
\end{lemma}

To prove this we take, for each $(C_i,C_j)$ non-uniform, some sets $C_{ij} \subset C_i$, $C_{ji} \subset C_j$ witnessing this so that
$$\left|C_{i j}\right| \geq \varepsilon\left|C_i\right|=\varepsilon c, \left|C_{j i}\right| \geq \varepsilon\left|C_j\right|=\varepsilon c$$
$$\left|d\left(C_{i j}, C_{j i}\right)-d\left(C_i, C_j\right)\right| \geq \varepsilon$$.

We would like to partition simultaneously all possible $C_{ij}$ into new sets $C_h$ so that each $C_{ij}$ is a union of a bunch of $C_h$'s, which would have larger energy, this isn't quite possible, but we can "atomise" each $C_i$ by making equivalent points not distinguishable by being in different $C_{ij}$s.
We then pick $H = 4^k - 2^{k-1}$ different $\floor{c/4^k}$-subsets, all contained in some atom of $C_i$.

This new partition can be shown to have an energy of at least $\epsilon ^5 /4 $ more.



































\end{document}
