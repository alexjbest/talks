%\documentclass[notheorems]{beamer}
%\documentclass[handout]{beamer}
%\documentclass[handout,notes=show]{beamer}

\usetheme{metropolis}
%\usecolortheme{dolphin}
% No navigation bars
\beamertemplatenavigationsymbolsempty

\usepackage{amsmath, amssymb, amsfonts, tikz}
\usepackage[utf8]{inputenc}
\usepackage[T1]{fontenc}
\usepackage[english]{babel}
\usepackage{tabularx}
\usepackage{mathtools}
\usepackage{xmpmulti}

\setsansfont[
    Extension      = .otf,
    UprightFont    = *-Light,
    ItalicFont     = *-LightItalic,
    BoldFont       = *-Regular,
    BoldItalicFont = *-RegularItalic
]{FiraSans}
\setmonofont[
    Extension   = .otf,
    UprightFont = *-Regular,
    BoldFont    = *-Medium
]{FiraMono}

\definecolor{Purple}{HTML}{911146}
\definecolor{Orange}{HTML}{CF4A30}
\definecolor{Tan}{RGB}{225,221,191}
\definecolor{Green}{RGB}{76,131,122}
\definecolor{DB}{RGB}{4,37,58}

% Theme colors are derived from these two elements
\setbeamercolor{alerted text}{fg=Green}

% ... however you can of course override styles of all elements
\setbeamercolor{frametitle}{bg=Tan,fg=DB}

\metroset{titleformat=smallcaps}


\usepackage{amsthm}
\theoremstyle{plain}
\newtheorem{theorem}{Theorem}[section]
\newtheorem{corollary}[theorem]{Corollary}
\newtheorem{lemma}[theorem]{Lemma}
\newtheorem{algorithm}[theorem]{Algorithm}
\newtheorem{proposition}[theorem]{Proposition}
\newtheorem{claim}[theorem]{Claim}
\newtheorem{fact}[theorem]{Fact}
\newtheorem{conjecture}[theorem]{Conjecture}
%% Definition-like environments, normal text
%% Numbering is in sync with theorems, etc
\theoremstyle{definition}
\newtheorem{definition}[theorem]{Definition}
%% Remark-like environments, normal text
%% Numbering is in sync with theorems, etc
\theoremstyle{definition}
\newtheorem{remark}[theorem]{Remark}
\newtheorem{observation}[theorem]{Observation}
%% Example-like environments, normal text
%% Numbering is in sync with theorems, etc
\theoremstyle{definition}
\newtheorem{example}[theorem]{Example}
\newtheorem{question}[theorem]{Question}
\newcommand{\terminology}[1]{\textbf{#1}}

\newcommand{\NN}{\mathbf{N}}
\newcommand{\ZZ}{\mathbf{Z}}
\newcommand{\QQ}{\mathbf Q}
\newcommand{\CC}{\mathbf C}
\newcommand{\RR}{\mathbf R}
\newcommand{\FF}{\mathbf F}
\newcommand{\lt}{<}
\newcommand{\gt}{>}
\newcommand{\amp}{&}
\newcommand{\diff}{\mathop{}\!\mathrm{d}}
\newcommand{\ints}{\mathcal{O}}
\newcommand{\ideal}[1]{\mathfrak{#1}}
\usepackage{mathrsfs}\usepackage{cancel}
\newcommand{\Gal}[2]{\operatorname{Gal}(#1/#2)}
\newcommand{\absgal}[1]{\operatorname{Gal}(\overline{#1}/#1)}
\DeclareMathOperator{\USp}{USp}
\DeclareMathOperator{\Spec}{Spec}

\newcommand{\sheaf}[1]{\operatorname{\mathcal{#1}}}
\newcommand{\inv}{^{-1}}
\DeclareMathOperator{\norm}{Nm}
\DeclareMathOperator{\ord}{ord}
\DeclareMathOperator{\divisor}{div}
\DeclareMathOperator{\PP}{\mathbf{P}}
\DeclareMathOperator{\Hom}{Hom}
\DeclareMathOperator{\Mat}{Mat}
\DeclareMathOperator{\End}{End}

\newcommand{\lb}{[}
\newcommand{\rb}{]}

\setbeamerfont*{subtitle}{size=\small,shape=\scshape}

\author{Alex J. Best}
\institute{Boston University}
\date{4/6/2019}
\title{Computing Coleman integrals on superelliptic curves}
\subtitle{Arithmetic of low dimensional abelian varieties -- ICERM}

\begin{document}

\begin{frame}
  \titlepage

  \note[item]{Thank the audience for being awake.}
\end{frame}

%\begin{frame}
%\frametitle{Table of Contents}
%\tableofcontents[currentsection]
%\end{frame}


\begin{frame}{Background}
    \alert{Coleman integration} is a $p$-adic integration theory that may be applied to integrate 1-forms on curves.

    \pause%
    Fix a curve $C/\ZZ_{p^n}$ with good reduction, a point $b\in C$, and $A$ the ring of (overconvergent) functions on $C$, defines $\int_b\colon \Omega^{1}_{A} \to A$,
    satisfying the usual properties (fundamental theorem of calculus, linearity, additivity in endpoints).

    \pause Many applications; finding rational points (Chabauty-Coleman(-Kim)), verifying torsion points on Jacobians of curves, defining regulators and period maps, \ldots


    \pause Since the work of Balakrishnan-Bradshaw-Kedlaya, we can compute Coleman integrals of $\{x^i\diff x/y\}_{i=0}^{2g-1}$ on hyperelliptic curves.

    \pause This algorithm took time proportional to $p$, as have extensions.

\end{frame}

\begin{frame}{Superelliptic curves and their Jacobians}
    \begin{theorem}
        Let \vspace{-13pt} $$ C/\ZZ_{p^n}\colon y^a = h(x)$$
        with $ \gcd(a,\deg(h)) = 1$, $ p\nmid a$, Let $ M$ be the matrix of Frobenius, acting on  $ H^1_\mathrm{dR}(C)$, basis $ {\{\omega_{i,j}= x^i\diff x /y^j\}}_{i=0,\ldots, b-2,j=1,\ldots,a}$, and points $ P,Q\in C(\QQ_{p^n})$ known to precision $ p^N$, if $ p \gt (aN - 1)b$, the vector of Coleman integrals $\left(\int_P^Q \omega_{i,j}\right)_{i,j}$ can be computed in time \vspace{-13pt}
        $$ \widetilde O\left(g^3 \sqrt{p}n N^{5/2} + N^4 g^4 n^2 \log p \right)$$
        to absolute precision  $ N - v_p(\det(M-I))$.
    \end{theorem}

    \pause%
    By integrating invariant differentials we can check/guess linear relations between points on the Jacobian of this superelliptic curve. \pause

    Speed of this algorithm may lend itself to answering distributional questions?

\end{frame}


%\begin{frame}{}
%In this lightning talk we discuss recent work to develop algorithms to compute these integrals specifically on superelliptic curves.
%This technique provides a quick way to find and verify torsion points on the Jacobians of such curves.
%\end{frame}

\end{document}
